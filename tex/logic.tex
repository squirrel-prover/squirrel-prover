\section{Abstract meta-logic}

As a first step we define a meta-logic that is abstract, in that it does
not explicitly refer to the protocol under study. It has all the ingredients
for reasoning about protocols (mainly indexed symbols and macro symbols)
but their interpretation is, for now, arbitrary.

\subsection{Base logic}

The base logic is that of Bana and Comon (CCS'14). It is first-order logic
with a distinguished kind of constant for representing names (written
$\mathsf{n}$, $\mathsf{m}$, $\mathsf{k}$, etc.)
and featuring a single predicate noted $\sim$ (or rather, a family
of predicates $\sim_k$ of arity $2 k$ for all $k\in\mathbb{N}$).
The terms are all of the same sort: they are meant to represent
messages\footnote{
  \david{I hope/believe that the distinction between booleans and messages
  can be ignored for the theory.}
}.

In practice, the logic will be considered with function symbols for
representing cryptographic primitives as well as attacker computations, and
boolean constants and connectives.

The formulas of the base logic are intended to be interpreted in
\emph{computational models}, where:
\begin{itemize}
  \item terms are interpreted as PPT Turing machines which,
    given infinite random tapes and a security parameter output a bitstring;
  \item $\sim$ is interpreted as indistinguishability;
  \item names are independent random samplings;
  \item function symbols correspond to deterministic machines.
\end{itemize}

In particular, function symbols corresponding to cryptographic primitives
are interpreted as implementations of the primitives, subject to some
assumptions.

For example, if $\mathsf{n}$ and $\mathsf{m}$ are distinct names,
the (atomic) formulas $\mathsf{n}\sim\mathsf{m}$ and
$\mathsf{n}\stackrel{.}{=}\mathsf{m}\sim\mathsf{false}$
are valid.

\subsection{Meta-logic}

Terms of the meta-logic are of three possible sorts: index, timestamps
or message. Formulas of the meta-logic may feature predicates over each sort.
Given interpretation $I$ and $T$ of index and timestamp variables
into two finite sets, a formula $\phi$ of the meta-logic will be translated
into a term $(\phi)^{I,T}$ of the base logic in which index and timestamps,
and predicates over them, will have disappeared. The validity of the
meta-logic formula $\phi$ will be defined as the validity of the base logic
$(\phi)^{I,T}\sim\mathsf{true}$.

\paragraph{Terms}

We assume three infinite sets of variables:
$\X$ (whose elements are noted $x$, $y$, $z$) for message variables;
$\I$ (whose elements are noted $i$, $j$, $k$) for index variables;
$\XT$ (whose elements are noted $\tau$) for timestamp variables.

We assume a set $\F$ of indexed function symbols
(used to model encryptions, pairs,\dots).
Each of these symbols comes with an index arity as well as an usual arity:
if $f\in\F$ has index arity $k$ and arity $n$,
then for all index variables $i_1,\ldots,i_k$ and terms $t_1,\ldots,t_n$,
$f[i_1,\ldots,i_k](t_1,\ldots,t_n)$ is a term.

We assume a set $\N$ of indexed names symbols,
(used to model random samplings)
and a set of indexed constants $\Actions$
(used to model specific timestamps).
These indexed symbols only have an index arity: they
cannot be applied to terms.

We finally assume a set $\M$ of indexed macro symbols
equipped with an index arity.
An indexed macro symbol will be applied to the specified number of
indices and also to a single timestamp.
This timestamp gives the instant at which the macro is evaluated.

We shall write $\F_k$ (resp.\ $\N_k$ or $\M_k$) for the set of function
symbols (resp.\ name or macro symbols) of index arity $k$.

\begin{figure}[h]
\[
  \begin{array}{rclll}
    i &:=& i & \text{index variable} & \\
    \\
    T &:=& \tau & \text{timestamp variable} & \\
      &\mid& a[i_1,\ldots,i_k] & \text{constant} & (a\in\mathbb{A}) \\
      &\mid & \pre(T) & \text{predecessor} & \\
    \\
    t & := & \mathsf{n}[i_1,\ldots,i_n] &\text{name} & (\mathsf{n}\in\N_k) \\
      & \mid & x  & \text{message variable} & (x\in\X) \\
      & \mid & f[i_1,\ldots,i_k](t_1,\dots,t_n) &\text{function application} & (f\in\F_k)\\
      & \mid & m[i_1,\ldots,i_k]@T &\text{macro application} & (m\in\M_k)\\
  \end{array}
\]
\caption{Syntax of meta-logic terms}\label{fig:terms}
\end{figure}

\begin{definition}
  Given a (meta logic) signature $\Sigma = (\F,\N,\M,\Actions)$
  and some sets of variables
  $\X$, $\I$ and $\XT$,
  \cref{fig:terms} defines the syntax of (meta-logic) terms
  of sort message (noted $t$) and timestamp (noted $T$).
  The only terms of sort index are index variables.
  The set of message terms of the meta-logic is noted $\Msg_\Sigma$.
\end{definition}

\begin{remark}
  Function symbols representing cryptographic primitives will have a usual
  arity k and as index arity $0$.
  Functions symbols representing identities (for example, a constant value
  associated to each tag) will have $0$ as usual arity and $1$ as index arity.
  Macro symbols will typically include $\mout$ and $\minp$
  (with index arity $0$) for representing messages inputted and
  outputted at a particular point of an execution.
  Macro symbols will also be used to represent states (memory cells, database),
  usually with index arity superior or equal to $1$.
\end{remark}


\paragraph{Formulas}

The syntax of the meta-logic formulas is given in \cref{fig:formulas}.
We could have included more generally a notion of predicate macro that
would have to be expanded when translating a formula from the meta-logic
to the base logic. Instead we opted to list the few atoms that we
will use in practice.

\begin{figure}[h]
  \[
  \begin{array}{c}
   \begin{array}{rcll}
    \T &  := & \timestamp \mid \idx & \text{(meta logic sorts)} \\
    \\[2ex]
   \atom & := & t=t'
 & \text{atomic proposition over messages } \\
  &\mid & T=T' \mid T \leq T' \mid \happens(T) &  \text{atomic proposition
  over timestamps } \\
  &\mid & i=i'  &  \text{atomic proposition
    over indices } \\
    \end{array}
\\
\\
     \begin{array}{rcll}
    \phi & ::= &  \true \mid \false \mid \phi \wedge \phi' \mid  \phi
    \vee \phi' \mid   \phi \Rightarrow \phi'\mid \neg \phi \mid
    \forall x : \T.\ \phi \mid \exists x:\T.\ \phi \mid \atom &
    \end{array}

\end{array}
    \]
    \caption{Syntax of meta-logic formulas}\label{fig:formulas}
\end{figure}
