\documentclass[a4paper]{article}

\usepackage{lipsum}
\usepackage[normalem]{ulem}
\usepackage{mathtools}
\usepackage{mathpartir}
\usepackage{stmaryrd}
\usepackage{amsthm}
\usepackage{float}
\usepackage{wrapfig}
\usepackage{framed}
\usepackage{xcolor}
\usepackage{textcomp}
\usepackage{xspace}
\usepackage{amsmath,amssymb,amsfonts}
\usepackage[T1]{fontenc}
\usepackage[utf8]{inputenc}
\usepackage{cite}
\usepackage{algorithmic}
\usepackage{graphicx}
\usepackage{placeins}
\usepackage[margin=3cm]{geometry}
\usepackage{multicol}

\usepackage{url}
\usepackage{tikz}
\usepackage{proof}
\usepackage{hyperref}
\usepackage[capitalize,nameinlink,noabbrev]{cleveref}
\usepackage{enumerate}

\usepackage{msc5}

\setlength{\ULdepth}{0.15em}

\usetikzlibrary{backgrounds}
\usetikzlibrary{intersections}
\usetikzlibrary{scopes}
\usetikzlibrary{calc}
\usetikzlibrary{decorations.pathreplacing}
\usetikzlibrary{decorations.pathmorphing,shapes}
\usetikzlibrary{backgrounds}
\usetikzlibrary{shapes.misc}

\newtheorem{assumption}{Assumption}
\newtheorem*{theorem*}{Theorem}
\newtheorem*{lemma*}{Lemma}
\newtheorem*{proposition*}{Proposition}



\newtheorem{definition}{Definition}
\newtheorem{example}{Example}
\newtheorem{proposition}{Proposition}
\newtheorem{theorem}{Theorem}
\newtheorem{lemma}{Lemma}
\newtheorem{corollary}{Corollary}

\theoremstyle{remark}
\newtheorem{remark}{Remark}

\newcommand{\david}[1]{\textcolor{orange}{(\textbf{David:} #1)}}
\newcommand{\stef}[1]{\textcolor{red}{(\textbf{Stéphanie:} #1)}}
\newcommand{\solene}[1]{\textcolor{teal}{(\textbf{Solène:} #1)}}
\newcommand{\adrien}[1]{\textcolor{blue}{(\textbf{Adrien:} #1)}}
\newcommand{\charlie}[1]{\textcolor{cyan}{(\textbf{Charlie:} #1)}}

\newcommand{\true}{\textsf{true}}
\newcommand{\false}{\textsf{false}}

\newcommand{\pvec}[1]{\vec{#1}\mkern2mu\vphantom{#1}}

\newcommand{\sfr}{\textsf{r}}
\newcommand{\sfx}{\textsf{x}}
\newcommand{\sfy}{\textsf{y}}
\newcommand{\sfz}{\textsf{z}}
\newcommand{\sfb}{\textsf{b}}

\newcommand{\bc}{\textsf{BC}}
\newcommand{\ra}{\rightarrow}
\newcommand{\mdp}{\mathbin{\|}}

\newcommand{\mdef}{\;\;\mathbin{:=}\;\;}

\newcommand{\cexists}{\exists}
\newcommand{\cforall}{\forall}

\newcommand{\judge}[3]{#1 \mdp #2 \vdash #3}
\newcommand{\subst}[3]{#1\{#2 \mapsto #3\}}
\newcommand{\substs}[2]{#1\{#2\}}

\newcommand{\thetap}{{\theta'}}
\newcommand{\thetapp}{{\theta''}}
\newcommand{\thetao}{{\theta_0}}
\newcommand{\thetat}{{\theta_1}}
\newcommand{\psip}{{\psi'}}
\newcommand{\deltap}{{\delta'}}
\newcommand{\gammap}{{\gamma'}}

\newcommand{\valpha}{\pvec{\alpha}}
\newcommand{\valphap}{\pvec{\alpha}'}
\newcommand{\valphao}{{\pvec{\alpha}_0}}
\newcommand{\valphat}{{\pvec{\alpha}_1}}
\newcommand{\valphatt}{{\pvec{\alpha}_2}}
\newcommand{\vbeta}{\pvec{\beta}}
\newcommand{\vbetap}{\pvec{\beta}'}
\newcommand{\vgamma}{\pvec{\gamma}}
\newcommand{\vgammap}{\pvec{\gamma}'}
\newcommand{\vdelta}{\pvec{\delta}}
\newcommand{\vdeltap}{\pvec{\delta}'}

\newcommand{\wf}{\textsf{well-formed}}

\newcommand{\env}{\textsf{E}}
\newcommand{\envp}{\textsf{E'}}

\newcommand{\pcnstr}{\mathcal{P}}
\newcommand{\cnstr}{\textsf{C}}

\newcommand{\act}{{\textsf{a}}}
\newcommand{\aset}{\mathcal{S}}

\newcommand{\decl}[3]{\left(#1 , #2 : #3\right)}
\newcommand{\decls}{\mathcal{D}}

\newcommand{\pvars}{\textsf{p-vars}}
\newcommand{\pvtype}[2]{#1 : #2}
\newcommand{\facts}[3]{\pvtype{#1}{#2} \mid #3}

\newcommand{\centail}[3]{#1 \mid #2 \vdash_{\textsf{c}} #3}
% Rules
\newcommand{\defunroll}[1]{\textsf{unroll}($#1$)}
\newcommand{\pvintro}{\textsf{pv-intro}}
\newcommand{\fintro}{\textsf{f-intro}}

\newcommand{\clstrength}{\textsf{c-left-strength}}
\newcommand{\crstrength}{\textsf{c-right-strength}}
\newcommand{\cempty}{\textsf{c-empty}}
\newcommand{\cdisj}{\textsf{c-disj}}

\newcommand{\ceelim}{\textsf{c-}$\exists$\textsf{-elim}}

\newcommand{\landelim}{$\wedge$\textsf{-left-elim}}
\newcommand{\lorelim}{$\vee$\textsf{-left-elim}}

\newcommand{\randelim}{$\wedge$\textsf{-right-elim}}


\newcommand{\apply}{\textsf{apply}}
\newcommand{\requ}{\textsf{r-equ}}


%%% Local Variables:
%%% mode: latex
%%% TeX-master: "main"
%%% End:


\begin{document}
\title{Meta-Logic for BC}

\author{David Baelde, Stéphanie Delaune,
  Charlie Jacomme, Adrien Koutsos, Solène Moreau}

\maketitle

\vfill

\tableofcontents

\vfill

Bana and Comon have introduced an approach, which they call
\emph{computationally complete symbolic attacker} (CCSA),
to formulate security properties in the computational model as first-order
logic formulas. Roughly, they translate a security property (for a finite set
of traces) as a first-order formula, then they seek to show that this formula
is valid in all \emph{computational models} where attacker computations
are unspecified and primitives satisfy some cryptographic assumptions. The
way this is done is by showing that the formula is a logical consequence of
some axioms, which are sound w.r.t.\ the considered class of computational
models.

We introduce here a \emph{meta-logic} whose formulas correspond to schemas
of formulas in the \emph{base logic}
--- which we might call the CCSA or BC logic.
We present inference rules which are sound in the same way as before,
i.e.\ any instance of the schema is a formula that is valid in all
computational models subject to the relevant cryptographic assumptions.

We define next the meta-logic which abstractly allows to reason about
executions of a protocol. The precise definition of protocols and their
semantics is not needed for that first step, and is defined only in a
second part.

\section{Abstract meta-logic}

As a first step we define a meta-logic that is abstract, in that it does
not explicitly refer to the protocol under study. It has all the ingredients
for reasoning about protocols (mainly indexed symbols and macro symbols)
but their interpretation is, for now, arbitrary.

\subsection{Base logic}

The base logic is that of Bana and Comon (CCS'14). It is first-order logic
with a distinguished kind of constant for representing names (written
$\mathsf{n}$, $\mathsf{m}$, $\mathsf{k}$, etc.)
and featuring a single predicate noted $\sim$ (or rather, a family
of predicates $\sim_k$ of arity $2 k$ for all $k\in\mathbb{N}$).
The terms are all of the same sort: they are meant to represent
messages\footnote{
  David: I hope/believe that the distinction between booleans and messages
  can be ignored for the theory.
}.

In practice, the logic will be considered with function symbols for
representing cryptographic primitives as well as attacker computations, and
boolean constants and connectives.

The formulas of the base logic are intended to be interpreted in
\emph{computational models}, where:
\begin{itemize}
  \item terms are interpreted as PPT Turing machines which,
    given infinite random tapes and a security parameter output a bitstring;
  \item $\sim$ is interpreted as indistinguishability;
  \item names are independent random samplings;
  \item function symbols correspond to deterministic machines.
\end{itemize}

In particular, function symbols corresponding to cryptographic primitives
are interpreted as implementations of the primitives, subject to some
assumptions.

For example, if $\mathsf{n}$ and $\mathsf{m}$ are distinct names,
the (atomic) formulas $\mathsf{n}\sim\mathsf{m}$ and
$\mathsf{n}\stackrel{.}{=}\mathsf{m}\sim\mathsf{false}$
are valid.

\subsection{Terms}

Terms of the meta-logic are of three possible sorts: index, timestamps
or message. Formulas of the meta-logic may feature predicates over each sort.
Given interpretation $I$ and $T$ of index and timestamp variables
into two finite sets, a formula $\phi$ of the meta-logic will be translated
into a term $(\phi)^{I,T}$ of the base logic in which index and action terms,
and predicates over them, will have disappeared. The validity of the
meta-logic formula $\phi$ will be defined as the validity of the base logic
$(\phi)^{I,T}\sim\mathsf{true}$.

\medskip

We assume three infinite sets of variables:
$\X$ (whose elements are noted $x$, $y$, $z$) for message variables;
$\I$ (whose elements are noted $i$, $j$, $k$) for index variables;
$\XT$ (whose elements are noted $\tau$) for timestamp variables.

We assume a set $\F$ of indexed function symbols
(used to model encryptions, pairs,\dots).
Each of these symbols comes with an index arity as well as an usual arity:
if $f\in\F$ has index arity $k$ and arity $n$,
then for all index variables $i_1,\ldots,i_k$ and terms $t_1,\ldots,t_n$,
$f[i_1,\ldots,i_k](t_1,\ldots,t_n)$ is a term.

We assume a set $\N$ of indexed names symbols,
(used to model the randomness used in a protocol)
and a sets of indexed constants $\Actions$
(used to model specific timestamps).
These indexed symbols only have an index arity: they
cannot be applied to terms.

We finally assume a set $\M$ of indexed macro symbols
equipped with an arity and index arity, like function symbols.
An indexed macro symbol will be applied to the specified number of
indices, variables, and also to a single timestamp.
We will typically assume macro symbols $\mout$ and $\minp$
(with index arity $0$) for representing messages inputted and outputted
at a particular point of an execution.

We shall write $\F_k$ (resp.\ $\N_k$ or $\M_k$) for the set of function
symbols (resp.\ name or macro symbols) of index arity $k$.

\begin{figure}
\[
  \begin{array}{rcll}
    n &:=& \mathsf{n}[i_1,\ldots,i_n] & (\mathsf{n}\in\N_k)
    \\
    F &:=& f[i_1,\ldots,i_k] & (f\in\F_k)
    \\
    M &:=& m[i_1,\ldots,i_k] & (m\in\M_k)
  \end{array}
\]
\[
  \begin{array}{rcll}
    T &:=& \tau & \text{timestamp variable} \\
      &\mid& a[i_1,\ldots,i_k] & \text{constant (for $a\in\mathbb{A}$)} \\
      &\mid & \pre(T) & \text{predecessor}
\end{array}
   \]
\[
     \begin{array}{rcll}
    t & := & n &\text{name} \\
    & \mid & x  & \text{message variable $x\in\X$} \\
    & \mid & F(t_1,\dots,t_n) &\text{function application}\\
    & \mid & M(t_1,\ldots,t_n)@T &\text{macro application}\\
       \end{array}
     \]
     \caption{Meta-logic terms}\label{fig:terms}
\end{figure}

\begin{definition}
  Given a signature $\Sigma = (\F,\N,\M,\Actions)$ and some sets of variables
  $\X$, $\I$ and $\XT$,
  \cref{fig:terms} defines the syntax of (meta-logic) terms
  of sort message (noted $t$) and timestamp (noted $T$).
  The only terms of sort index are index variables.
  The set of message terms of the meta-logic is noted $\Msg_\Sigma$.
\end{definition}

We can now define a meta-interpretation as the structure needed to
give a meaning to index and timestamp terms. A meta-interpretation
also induces a translation from a meta-logic signature $\Sigma$
to some base logic signature $\Sigma^I$. For example, if
$\Sigma$ contains a name $\mathsf{n}$ with index arity $1$, and
the meta-interpretation interprets indices in a domain $D_\I = \{
  17, 22 \}$, the base logic signature will feature
two names $\mathsf{n}_{17}$ and $\mathsf{n}_{22}$.

\begin{definition}
  A meta-interpretation $I$ for $\Sigma = (\F,\M,\N)$ consists in:
  \begin{itemize}
    \item two finite sets $D_\I$ and $D_\XT$ called the index and timestamp
      domains of the interpretation;
    \item mappings $\sigma_\I : \I \to D_\I$
      and $\sigma_\XT : \XT \to D_\XT$ that interpret index and
      timestamp variables as elements of their respective domains;
    \item a total ordering $\leq$ over $D_\XT$,
      a function $p : D_\XT \to D_\XT$ (for interpreting the
      predecessor\footnote{
        It does not matter that the predecessor means anything
        wrt.\ the ordering. We will impose later that it behaves
        well enough.
      })
      and a subset $H_\XT\in D_\XT$ (for identifying timestamps
      that actuallly happen in an execution\footnote{
        We cannot identify timestamps freely in the meta-interpretation
        because we want to be able to use axioms such as
        $\forall i\neq j.~ a[i] \neq a[j]$ without restricting
        them to timestamps that really happen.
      });
    \item for each constant $a \in \Actions$ of index arity $k$,
      an interpretation $\hat{a} : D_\I^k \to D_\XT$;
    \item for each macro symbol $m \in \M$ of index
      arity $k$ and arity $n$, an interpretation
      $\hat{m} : D_\I^k \times \Msg_\Sigma^n \times D_\XT \to \Msg^I_\Sigma$,
      where $\Msg^I_\Sigma$ is the set of terms of the base
      logic for the signature
      $(\F^I,\N^I)$ with
      $$ \F^I =
      \{ f_{e_1,\ldots,e_k} : f \in \F_k, e_1,\ldots,e_k \in D_I \}
      \text{ and }
      \N^I = \{ \mathsf{n}_{e_1,\ldots,e_k} : \mathsf{n}\in\N_k,
      e_1,\ldots,e_k\in D_I \}.$$
  \end{itemize}
\end{definition}

Note that, for each name $\mathsf{n}\in \N$ and indices $n_1,\ldots,n_k \in
D_\I$, $\mathsf{n}_{i_1,\ldots,i_k}$ refers to a distinct name in $\N^I$.

\begin{definition}
  Given a meta-interpretation $I$
  we define $(T)^{I} \in D_\XT$ and $(t)^{I} \in \Msg^I_\Sigma$ as follows:
  \begin{eqnarray*}
    (\tau)^{I} &=& \sigma_\XT(\tau) \\
    (\pre(T))^{I} &=& p((T)^{I}) \\
    (a[i_1,\ldots,i_k])^{I} &=& \hat{a}(\sigma_\I(i_1),\ldots,\sigma_\I(i_k))
  \end{eqnarray*}
  \begin{eqnarray*}
    (\mathsf{n}[i_1,\ldots,i_k])^{I} &=& \mathsf{n}_{\sigma_\I(i_1),\ldots,\sigma_\I(i_k)}
    \\
    (x)^{I} &=& x
    \\
    (f[i_1,\ldots,i_k](t_1,\ldots,t_n))^{I} &=&
    f_{\sigma_\I(i_1),\ldots,\sigma_\I(i_k)}\bigl(
      (t_1)^{I},\ldots,(t_n)^{I}
    \bigr)
    \\
    (m[i_1,\ldots,i_k](t_1,\ldots,t_n)@T)^{I} &=&
    \hat{m}(\sigma_\I(i_1),\ldots,\sigma_\I(i_k),
      (t_1)^{I},\ldots,(t_n)^{I},
      (T)^{I})
  \end{eqnarray*}
\end{definition}

\begin{example}
  Consider the meta-logic term $t := \mathsf{h}(\mout@a[i],\mathsf{k}[i])$
  and an interpretation $I$ with $D_T = [1;10]$ and $D_I = [1;3]$
  (which might be relevant if we are considering traces of ten actions
  with three agents) such that $\sigma_I(i)=2$ and $\hat{a}(2)=10$
  and $\hat{\mout}(10)=\mathsf{ok}$ (the message outputted at step 10
  is $\mathsf{ok}$).
  We then have $(t)^I = \mathsf{h}(\mathsf{ok},\mathsf{k}_2)$.
\end{example}

The reason why macros take only one timestamp argument is purely practical:
we have no use for more. At this point one might wonder why we separate
indices and actions given that they are interpreted similarly: it is again
purely practical, we will use them for different purposes, and we will need
less structure on indices than on actions, making reasoning easier on them
(one can simply compare indices, there is no ordering and no predecessor
operation on them).

\subsection{Formulas}

The syntax of the meta-logic formulas is given in \cref{fig:syntax}.
We could have included more generally a notion of predicate macro that
would have to be expanded when translating a formula from the meta-logic
to the base logic. Instead we opted to list the few atoms that we
will use in practice.

\begin{figure}
  \[
  \begin{array}{c}
   \begin{array}{rcll}
    \T &  := & \timestamp \mid \idx & \text{(meta logic sorts)} \\
    \\[2ex]
   \atom & := & t=t'
 & \text{atomic proposition over messages } \\
  &\mid & T=T' \mid T \leq T' \mid \happens(T) &  \text{atomic proposition
  over timestamps } \\
  &\mid & i=i'  &  \text{atomic proposition
    over indices } \\
    \end{array}
\\
\\
     \begin{array}{rcll}
    \phi & ::= &  \true \mid \false \mid \phi \wedge \phi' \mid  \phi
    \vee \phi' \mid   \phi \Rightarrow \phi'\mid \neg \phi \mid
    \forall x : \T.\ \phi \mid \exists x:\T.\ \phi \mid \atom &
    \end{array}

\end{array}
    \]
    \caption{Syntax of first order formulas.}\label{fig:syntax}
\end{figure}

\begin{definition}
  If $I$ is a meta-interpretation and $e\in D_\I$,
  $I[i\mapsto e]$ is the interpretation where $\sigma_\I$ is
  modified so that $\sigma_{\I}(i)=e$.
  We define the translation of a meta-logic formula $\phi$
  into the base logic \emph{term} $(\phi)^I$
  as follows:
  \begin{eqnarray*}
    (\phi\wedge\phi')^I &=& (\phi)^I \stackrel{.}{\wedge} (\phi')^I
    \quad \text{and similarly for other boolean connectives} \\
    (\forall i:\idx. \phi)^I &=&
    \stackrel{.}{\wedge}_{e\in D_\I} (\phi)^{I[i\mapsto e]} \\
    (\forall \tau:\timestamp. \phi)^I &=&
    \stackrel{.}{\wedge}_{e\in D_\XT} (\phi)^{I[\tau\mapsto e]} \\
    (\exists i:\idx. \phi)^I &=&
    \stackrel{.}{\vee}_{e\in D_\I} (\phi)^{I[i\mapsto e]} \\
    (\exists \tau:\timestamp. \phi)^I &=&
    \stackrel{.}{\vee}_{e\in D_\XT} (\phi)^{I[\tau\mapsto e]} \\
    (i=i')^I &=&
    \left\{\begin{array}{ll}
      \mathsf{true} & \text{if } \sigma_\I(i)=\sigma_\I(i') \\
      \mathsf{false} & \text{otherwise}
    \end{array}\right. \\
    (T = T')^I &=&
    \left\{\begin{array}{ll}
      \mathsf{true} & \text{if } (T)^I = (T')^I \\
      \mathsf{false} & \text{otherwise}
    \end{array}\right.
    \quad\text{ and similarly for $\leq$} \\
    (\happens(T))^I &=& \mathsf{true}
    \text{ if } (T)^I \in H_\XT
    \text{ and } \mathsf{false} \text{ otherwise}
    \\
    (t=t')^I &=& (t)^I = (t')^I
  \end{eqnarray*}
\end{definition}

\begin{definition}
  A formula $\phi$ of the meta-logic is said to be valid when,
  for any meta-interpretation $I$, the base logic formula
  $(\phi)^I \sim \mathsf{true}$ is valid.
  (In other words, we have $\M,\sigma\models(\phi)^I\sim\mathsf{true}$
  i.e. the boolean term $(\phi)^I$ is true with overwhelming
  probability in any computational model $\M$ and for any interpretation
  $\sigma$ of the free message variables\footnote{
    In the tool, we only allow universal quantification over messages, and
  only allow it at toplevel. The validity of such formulas is the same as
  when the variables are left free.}.)
\end{definition}

\begin{definition}
  A meta-logic formula $\phi$ is a
  logical consequence of a set $S$ of meta-logic formulas
  (noted $S \models \phi$)
  when
  $\M\models(\phi)^I\sim\mathsf{true}$ holds for any $\M$ and $I$ such that
  $\M\models(\psi)^I\sim\mathsf{true}$ holds for all $\psi\in S$.
\end{definition}

We want to verify that a formula $\phi$ (the security property)
holds for all executions of some protocol, in all computational models
satisfying some cryptographic assumptions.
This can be guaranteed if we have $Ax\models\phi$
where $Ax$ is a (recursive) set of axioms that are sound
wrt.\ the intended class of meta-interpretations and computational models.
We might be able to formulate such a set $Ax$ (which would consist of
two parts, first the usual BC axioms, second some meta-logic axioms
constraining the meta-interpretations to correspond to protocol executions).
However, we will go for an easier task: designing a set of inference
rules that allow to derive meta-logic formulas (rather, sequents)
that are valid in all intended models.

\subsection{Sequent calculus}

In the tool, sequents come with an environment which explicitly
declares all the variables that might occur free in the sequent's formulas.
For simplicity, we do not include them here. We shall write $\vdash t:T$
when $t$ is a term of sort $\T$ (which might be $\idx$ or $\timestamp$).

\begin{definition}
  A sequent $\Gamma \vdash \phi$ is composed of a set of meta-logic formulas
  $\Gamma$ and a meta-logic formula $\phi$.
  It is valid when the meta-logic formula
  $(\wedge\Gamma) \Rightarrow \phi$ is valid.
\end{definition}

\begin{proposition}
  The rules of \cref{fig:lk,fig:names} are sound: if the premisses are valid,
  then so is the conclusion.
\end{proposition}

\begin{figure}
  \begin{mathpar}
    \inferrule[Axiom]{~}{\Gamma,\phi\vdash\phi}
    \quad\quad
    \inferrule[Cut]{
      \Gamma \vdash \phi
      \quad
      \Gamma,\phi \vdash \psi
    }{
      \Gamma \vdash \psi
    }
  \end{mathpar}
  \begin{mathpar}
  \inferrule[${\wedge}$-L]{
    \Gamma,\phi,\phi'\vdash\psi
  }{
    \Gamma,\phi\wedge\phi'\vdash\psi
  }
  \quad\quad
  \inferrule[${\wedge}$-R]{
    \Gamma \vdash \phi
    \quad
    \Gamma \vdash \phi'
  }{
    \Gamma \vdash \phi\wedge\phi'
  }
  \end{mathpar}
  \begin{mathpar}
    \inferrule[${\lnot}$-R]{
      \Gamma, \phi \vdash \bot
    }{
      \Gamma \vdash \lnot\phi
    }
    \quad\quad
    \inferrule[${\lnot}$-L]{
      \Gamma \vdash \phi
    }{
      \Gamma,\lnot\phi \vdash \psi
    }
    \quad\quad
    \inferrule[Raa]{
      \Gamma, \lnot\phi \vdash \bot
    }{
      \Gamma \vdash \phi
    }
  \end{mathpar}
  \begin{center}
  \emph{other propositional rules of classical sequent calculus}
  \end{center}
  \begin{mathpar}
      \inferrule[${=}$-R]{~}{\Gamma \vdash t=t} \quad\quad
      \inferrule[${=}$-L]{
        (\Gamma\vdash\phi)\{x\mapsto t',x'\mapsto t\}
      }{
        t=t', \Gamma\{x\mapsto t,x'\mapsto t'\} \vdash
        \phi\{x\mapsto t,x'\mapsto t'\}
      }
  \end{mathpar}
  \begin{mathpar}
     \inferrule[$\forall$-L]{
       \Gamma,\phi\{x\mapsto t\} \vdash \psi
       \quad
       \vdash t:\T
     }{
       \Gamma,\forall x:\T.\phi \vdash \psi}
     \quad\quad
     \inferrule[$\forall$-R]{
       \Gamma \vdash \phi}{\Gamma \vdash \forall x:\T:\phi}
  \end{mathpar}
   \caption{Generic inference rules. These are the rules of classical
   first-order sequent calculus. In \textsc{$\forall$-R} we require that
   $x$ does not appear free in $\Gamma$.}
   \label{fig:lk}
\end{figure}

\begin{figure}
  \begin{mathpar}
  \inferrule{
    \mathsf{n}\neq\mathsf{m}
  }{
    \Gamma,\mathsf{n}[\vec i]=\mathsf{m}[\vec j]\vdash \phi
  }
  \quad\quad
  \inferrule{
    \Gamma, i_1=j_1, \ldots, i_k=j_k \vdash \phi
  }{
    \Gamma,\mathsf{n}[i_1,\ldots,i_k]=\mathsf{n}[j_1,\ldots,j_k]\vdash \phi
  }
  \end{mathpar}
  \begin{mathpar}
    \inferrule{~}{
      \Gamma,t=\mathsf{n}[\vec i] \vdash \phi
    }
  \end{mathpar}
  \caption{Inference rules for equalities on names.
  The last rule only applies when, for any meta-interpretation $I$,
  the term $(t)^I$ does not contain any message variable and does
  not contain any occurrence of $\mathsf{n}_{\sigma_\I(\vec i)}$
  --- this condition will be over-approximated in our implementation.}
  \label{fig:names}
\end{figure}

\begin{proposition}
  The rules of \cref{fig:xor} are sound wrt.\ computational models
  where $\oplus$ is interpreted as exclusive or:
  if the premisses are valid in all such computational models,
  then so is the conclusion.
  \emph{Do we need to impose a condition on message lengths?}
\end{proposition}

\begin{figure}
  \begin{mathpar}
    \inferrule{~}{\Gamma \vdash t\oplus t = 0}
    \quad\quad
    \inferrule{~}{\Gamma \vdash t\oplus t' = t'\oplus t}
    \quad\quad
    \inferrule{~}{\Gamma \vdash t\oplus (t'\oplus t'') = (t\oplus t')\oplus t''}
  \end{mathpar}
  \begin{mathpar}
    \inferrule{
      (\Gamma \vdash \phi)\{x\mapsto \mathsf{m}[\vec j]\}
    }{
      (\Gamma \vdash \phi)\{x\mapsto t\oplus\mathsf{n}[\vec i]\}
    }
  \end{mathpar}
  \caption{Inference rules for exclusive or.
  The tool does not follow these rules closely, but uses a more high-level
  congruence closure algorithm modulo xor. The last rule is not implementated
  at all and is here mostly for illustration purposes, and is subject to
  the condition that, for any meta-interpretation $I$,
  $(\Gamma\vdash\phi)^I$ does not contain instances of the names
  $\mathsf{n}$ and $\mathsf{m}$ and does not contain message variables.
  This condition would be over-approximated in an implementation.}
  \label{fig:xor}
\end{figure}

%%% Local Variables:
%%% mode: latex
%%% TeX-master: "main"
%%% End:


\section{Protocols and trace logic}

Without entering into the details of the protocol semantics, we can already
restrict the set of meta-interpretations of interest, to justify some of the
rules that we use. We would alternatively impose these conditions from the
beginning in the definition of meta-interpretations.

\begin{definition}
A meta-interpretation is trace-like when:
\begin{itemize}
\item for every $t,t''\in H_\XT$ and $t'\in D_\XT$
  such that $t<t'<t''$, we have $t'\in H_\XT$;
\item for every $t\in H_\XT$, $p(t) < t$ and there
is no $t'\in D_\XT$ such that $p(t)<t'<t$;
\item for any two constants $a,b\in\Actions$,
  the images of $\hat{a}$ and $\hat{b}$ are disjoint,
  and $\hat{a}$ is injective;
\item probably some more\ldots
\end{itemize}
\end{definition}

\emph{[David] The current constraint solver is using more than these
  axioms, and is in fact inconsistent with the finiteness of $D_\XT$,
  cf. issue 22.
  For example, our axioms do not allow to derive a contradiction from
  $p(\tau)=\tau$, unless we know $\happens(\tau)$.}

\begin{proposition}
The rules of \cref{fig:tracelike} (\emph{TODO})
are sound wrt.\ trace-like meta-interpretations.
\end{proposition}

It is important that any protocol execution (to be defined next) can be
mapped to a trace-like meta-interpretation. This would not be possible,
for instance, if we had only one point outside of $H_\XT$.

\emph{[David] I have not carefully worked through what follows, and it
probably needs (at least) to be updated following the changes in the previous
section.}

\subsection{Protocols definitions}
We  model protocols as a set of possible actions available to the
attacker. An action essentially models a step of the protocol, where
the attacker provides some inputs, some conditions are checked, some
updates are performed, and finally an output is performed.

Some actions may be available to the attacker in parallel, or one
after the other. A protocol will then be a tree, each node labelled by an action, a parallel operator or a choice operator.

\begin{definition}
An action $s(\ov{id})$ is a tuple $(\phi,U,o)$, where $\ov{id} \in I^k$ is a list of indices variable, $\phi$ is a propositional formula over ground terms (called the condition of $s$), $U$ is a set of states updates (a mapping from state symbols to terms), and $o$ is a list of ground terms.
All indices appearing in $\phi,U,o$ must appear in $I$.

Given a set of actions $\Actions$, a protocol $P$ is a term built over the signature $\{\|/2,\oplus/2\} \cup \{a/1 \mid a \in \Actions\}$, such that when going down inside the protocol, the indices set of the actions can only grow. We denote $\Actions(P)$ the set of actions appearing inside $P$. We denote $\suc^*(a)$ all the descendants of a node (including itself), $\pre(a)$ the predecessor (ancestor) of $a$ inside $P$ and $\bro(a)$ its brother, which are defined only if they exists.
\end{definition}.

We may denote an action $s(\ov{id})$ to give explicitely its set of indexes.
For concision, we will denote protocols using $;,\|,\oplus$. For instance, $s_1; (s_2 \| s_3(id))$ represents the protocol where the attacker can first execute $s_1$, and can then execute in any order $s_2$, and an arbitrary number of occurences of $s_3(id)$.

There exists a natural translation from applied pi-calculus to this notion of actions, reminescent of the translation inside Horn Clauses performed by Proverif, or the translation inside Multi Set Rewritting rules performed by Sapic (a Tamarin extension).

\subsection{Symbolic protocol execution}

Protocols can be executed according to the ordering constraints on protocol steps defined by the protocol tree.
Each action depending on indices may be instantiated an arbitrary number of time using distinct indices. The protocol defines how the protocol steps may be executed. For instance, if some protocol step appears in
the trace with some defined indices, all its ancestors must have been executed previously in the trace.


\begin{definition}
  Given a protocol $P$, a symbolic trace of of $P$ is given, for any $n\in\mathbb{N}$, by a sequence of actions $a_0 (\ov{id}_0),\dots,a_n (\ov{id}_n)$, such that for all $1 \leq i,j \leq n$,
  \begin{itemize}
    \item each action is an instantiation of a protocol action and indices are uniformly instantiated, i.e, there exists $\sigma : \I \mapsto \I$, such that $a_i (\sigma (\ov{id})) \in\Actions(P)$,
    \item each action occurs only once, i.e, $a_i \neq{} a_j$ or $\ov{id}_i \neq\ov{id}_j$,
    \item if an action must happen before another one, it is reflected in the trace, i.e
      if $\pre (a_i(\sigma(\ov{id}_i)) \in \Actions(P)$, then there exists $k$ such that $\pre(a) = a_k(\sigma(\ov{id}_k))$,
    \item if the protocol enforces a choice between two actions, then it is respected in the trace, i.e if $\pre(a_i(\sigma(\ov{id}_i))) = \oplus$, then $a_j(\sigma(\ov{id}_j)) \notin \suc^*(\bro (a_i(\sigma(\ov{id}_i))))$,
      \item the actions only depends on previous actions, i.e, for any $\mout@a$ (resp. $\minp@a$) which appears in the terms of $a_i(\ov{id}_i)$, then there exists $k < i$ (resp. $k \leq i$) such that $a = a_k(\ov{id}_k)$.
      \end{itemize}
\end{definition}
Intuitively, a symbolic trace represent a possible scheduling of the protocol, when abstracting all the inputs. Some symbolic trace may not correspond to any actual execution of the protocol, if for instance an action require a condition which is never met.

Notice that the number of symbolic traces is infinite as soon as there exists an accessible action with a non nil set of indices.


\subsection{Concrete protocol execution}
As in the \BC model, we model arbitrary messages produced by the attacker using free function symbols $\G = \{g_i | i \in \mathbb{N}\} $, and extend the definition of a computational model $\M$ so that it contains Turing Machines for each attacker symbol.

Given a symbolic trace of the protocol, we can define the substitution which maps inputs, outputs and states to terms using the attacker function symbols, with the required attacker knowledge at each step.

\begin{definition}
  Given a symbolic trace $A = a_0,\dots,a_n$ of $P$, we define inductively $\sigma$ the instantiation of the inputs and ouputs of $A$ such that:
  \begin{itemize}
  \item $\minp@a_i\sigma := g_i( \mout@a_0\sigma,\dots,\mout@a_{i-1}\sigma  ) $
  \item $\mout@a_i\sigma := o_0\sigma, \dots, o_k\sigma$, where $o_0,\dots,o_k$ is the set of ouputs of  $a_i$,
    \item $s@a_i := u\sigma$, where $s$ is a state and $s \mapsto u$ is the latest update corresponding to $s$ appearing inside the actions before $a_i$.
\end{itemize}
\end{definition}
Note that $\sigma$ can indeed be defined, as by definition of a symbolic traces, the outputs of the action $a_i$ only depends on input values of previous actions.

\begin{definition}
  A (concrete) trace of $P$ is given by a symbolic trace $A = a_0,\dots,a_n$ of $P$ along with a computational model $\M$, such that with $\sigma$ the instantiation of $A$, we have that $A,\I,\M,\sigma \models \bigwedge_{1\leq i \leq n} \phi_i$, where $\phi_i$ is the condition of $a_i$.

  We denote $\tr(P)$ the set of traces of $P$.
\end{definition}

A trace of $P$ then corresponds to an actual possible evaluation of $P$. More strongly, $\tr(P)$ represents the set of all possible executions of the protocol $P$ for all possible probabilistic polynomial time attackers.

\subsection{Semantics of formulas for protocols}

\begin{definition}
  Let $\phi$ be a formula and $(A=(a_0,\dots,a_n),\M)$ a trace of $P$. If we denote $I(A,\phi)$ the set of indices appearing in $A$ extended with fresh indices for each quantification over indices in $\phi$ , and $\sigma$ the instantiation of $A$,

we say that $P$ satisfies $\phi$, denoted $P \models \phi$, if for all $(A,\M)$ in $\tr(P)$, $A, I(A,\phi), \M, \sigma \models \phi$.

\end{definition}

This semantics does match the intuition: for all possible executions of the protocol and for any

security parameter, the execution satisfies the formula.

$I(A,\phi)$ must depend both on the formula and the trace, so that the formula $\exists i:\idx, i=i$ is true, even when considering the empty protocol.


\section{Semantics}

We can now define a meta-interpretation as the structure needed to
give a meaning to index and timestamp terms. A meta-interpretation
also induces a translation from a meta-logic signature $\Sigma$
to some base logic signature $\Sigma^I$. For example, if
$\Sigma$ contains a name $\mathsf{n}$ with index arity $1$, and
the meta-interpretation interprets indices in a domain $D_\I = \{
  17, 22 \}$, the base logic signature will feature
two names $\mathsf{n}_{17}$ and $\mathsf{n}_{22}$.

As in the \BC model, we model arbitrary messages produced by the attacker
using free function symbols $\G = \{g_i | i \in \mathbb{N}\}$. We assume
from now on that our signature contains such symbols.

\paragraph{Meta-interpretation of a trace}

We construct, for each symbolic trace, a meta-interpretation representing
the trace. Some details of this definition will be understood in the next
section, where we identify some axioms that hold in all such
meta-interpretations.

\begin{definition}
  Given an abstract trace $\tr = \alpha_1\ldots\alpha_n$,
  we define the meta-interpretation $I_\tr$ as follows:
  \begin{itemize}
    \item an index domain $\D_\I$, which is the set of indices that occur in $\tr$;
    \item a timestamp domain $\D_\XT = \{ t_0 \} \uplus \{ a(\ov{n}) : a\in\Actions_k,
      \ov{n}\in\D_\I^k \}$;
    \item arbitrary mappings $\sigma_\I : \I \to D_\I$
      and $\sigma_\XT : \XT \to D_\XT$ that interpret index and
      timestamp variables as elements of their respective domains;
    \item a total ordering $\leq$, such that $t_0\leq t$ for all $t\in\D_\XT$,
      and $\alpha\leq\beta$ if $\alpha$ occurs before $\beta$ in $\tr$;
    \item a predecessor function $p$ mapping $t_0$ to itself and all
      other timestamps to their predecessor according to $\leq$;
    \item a subset $H_\XT\subseteq D_\XT$  of actions that occur in $\tr$,
      i.e.\ $H_\XT=\{\alpha_i : i\in[1;n]\}$;
    \item for each constant $a \in \Actions$ of index arity $k$,
    an interpretation $\hat{a} : D_\I^k \to D_\XT$ such that
    $\hat{a}(\ov{n}) = a(\ov{n})$;
    \item for each macro symbol $m \in \M$ of index
      arity $k$ and arity $n$, an interpretation
      $\hat{m} : D_\I^k \times \Msg_\Sigma^n \times D_\XT \to \Msg^{I_\tr}_\Sigma$,
      where $\Msg^{I_\tr}_\Sigma$ is the set of terms of the base
      logic for the (base logic) signature
      $(\F^{I_\tr},\N^{I_\tr})$ with
      $$\F^{I_\tr} =
        \{ f_{e_1,\ldots,e_k} : f \in \F_k, e_1,\ldots,e_k \in D_\I \}
        \text{ and }
        \N^{I_\tr} = \{ \mathsf{n}_{e_1,\ldots,e_k} : \mathsf{n}\in\N_k,
        e_1,\ldots,e_k\in D_\I \}.$$
      \footnote{\adrien{The interpretation of macros can depend on the precise element of $\D_\T$ it is interpreted at. In the tool, this is more restricted. E.g.\ $\hat{m}@a[i]$ can only depend on $a$, not on $i$  (except to re-use $i$ in the term). Instead, we could interpret macro symbol as terms of the meta-logic (with restrictions to have termination).}}
    \begin{itemize}
      \item $\hat{\minp}$ and $\hat{\mout}$ have arbitrary values on $t_0$
      \item \solene{$\hat{s}(t_0) =$ ???
      ---
      I do not know what should I write here, because the value we give to
      $\hat{s}(t_0)$ must be a term of the base logic for the (base logic)
      signature $(\F^{I_\tr},\N^{I_\tr})$.
      But the signature $(\F^{I_\tr},\N^{I_\tr})$ is not known at the time
      we define a protocol (which seems the appropriate place to give the
      initialisation parameters of each memory cells).}
      \item otherwise, $\hat{\minp}$, $\hat{\mout}$ and $\hat{s}$
      are uniquely defined by the following
      conditions, for all $i\in[1;n]$:
      $$\hat{\minp}(\alpha_i) =
      g_{i}(\hat{\mout}(\alpha_1),\ldots,\hat{\mout}(\alpha_{i-1}))$$
      $$\hat{\mout}(\alpha_i) = o_{\alpha_i}
      \{x_{\alpha_j}\mapsto\hat{\minp}(\alpha_j)\}_{j \leq i}
      \{x^s_{\alpha_i}\mapsto\hat{s}(\alpha_i)\}$$
      $$\hat{s}(\alpha_i) = u_{\alpha_{i-1}}^{s}
      \{x_{\alpha_j}\mapsto\hat{\minp}(\alpha_j)\}_{j \leq i-1}\{
      x^s_{\alpha_{i-1}}\mapsto\hat{s}(\alpha_{i-1})\}$$
    \end{itemize}
  \end{itemize}
\end{definition}

Note that, for each name $\mathsf{n}\in \N$ and indices $n_1,\ldots,n_k \in
D_\I$, $\mathsf{n}_{n_1,\ldots,n_k}$ refers to a distinct name in $\N^{I_\tr}$.
\adrien{The same is true for function symbols. The thing to remark here is that in a BC computational model, different names will always be interpreted by i.i.d.\ uniform random samplings. This is not true for function symbols ($\ne$ function symbols may have the same interpretation).}

% \begin{definition}
%   A meta-interpretation $I$\footnote{\adrien{I don't like the name meta-interpretation. Meta-model would be better. Or something like trace-model. }}
%   for $\Sigma = (\F,\M,\N,\Actions)$ consists in:
%   \begin{itemize}
%     \item two finite sets $D_\I$ and $D_\XT$ called the index and timestamp
%       domains of the interpretation;
%     \item mappings $\sigma_\I : \I \to D_\I$
%       and $\sigma_\XT : \XT \to D_\XT$ that interpret index and
%       timestamp variables as elements of their respective domains;
%     \item a total ordering $\leq$ over $D_\XT$,
%       a function $p : D_\XT \to D_\XT$ (for interpreting the
%       predecessor\footnote{
%         It does not matter that the predecessor means anything
%         wrt.\ the ordering. We will impose later that it behaves
%         well enough.
%       })
%       and a subset $H_\XT\subseteq D_\XT$ (for identifying timestamps
%       that actuallly happen in an execution\footnote{
%         We cannot identify timestamps freely in the meta-interpretation
%         because we want to be able to use axioms such as
%         $\forall i\neq j.~ a[i] \neq a[j]$ without restricting
%         them to timestamps that really happen.
%         \adrien{I think there are more fundamental reason for that. Considering the way for formulate things, for any $a \in \bbA$ (of arity 1) and $i \in D_\I$, $a[i]$ has an interpretation. But the protocol may not allow for it (because of restrictions). By consequence we need to be able to interpret $a[i]$ without it being valid.}
%       });
%     \item for each constant $a \in \Actions$ of index arity $k$,
%       an interpretation $\hat{a} : D_\I^k \to D_\XT$;
%     \item for each macro symbol $m \in \M$ of index
%       arity $k$ and arity $n$, an interpretation
%       $\hat{m} : D_\I^k \times \Msg_\Sigma^n \times D_\XT \to \Msg^I_\Sigma$,
%       where $\Msg^I_\Sigma$ is the set of terms of the base
%       logic for the (base logic) signature
%       $(\F^I,\N^I)$ with
%       \[
%         \F^I =
%         \{ f_{e_1,\ldots,e_k} : f \in \F_k, e_1,\ldots,e_k \in D_\I \}
%         \text{ and }
%         \N^I = \{ \mathsf{n}_{e_1,\ldots,e_k} : \mathsf{n}\in\N_k,
%         e_1,\ldots,e_k\in D_\I \}.
%       \]
%       \adrien{The interpretation of macros can depend on the precise element of $\D_\T$ it is interpreted at. In the tool, this is more restricted. E.g.\ $\hat{m}@a[i]$ can only depend on $a$, not on $i$  (except to re-use $i$ in the term). Instead, we could interpret macro symbol as terms of the meta-logic (with restrictions to have termination).}
%   \end{itemize}
% \end{definition}


\paragraph{Meta-interpretation of terms and formulas with relation to a trace}

Given a meta-interpretation $I_\tr$ for a given trace, we define the
interpretation in $I_\tr$ of terms of the meta-logic as terms of the base logic.

\begin{definition}
  Given a meta-interpretation $I_\tr$
  we define $(T)^{I_\tr} \in D_\XT$ and $(t)^{I_\tr} \in \Msg^{I_\tr}_\Sigma$ as follows:
  \begin{eqnarray*}
    (\tau)^{I_\tr} &=& \sigma_\XT(\tau) \\
    (\pre(T))^{I_\tr} &=& p((T)^{I_\tr}) \\
    (a[i_1,\ldots,i_k])^{I_\tr} &=& \hat{a}(\sigma_\I(i_1),\ldots,\sigma_\I(i_k))
  \end{eqnarray*}
  \begin{eqnarray*}
    (\mathsf{n}[i_1,\ldots,i_k])^{I_\tr} &=& \mathsf{n}_{\sigma_\I(i_1),\ldots,\sigma_\I(i_k)}
    \\
    (x)^{I_\tr} &=& x
    \\
    (f[i_1,\ldots,i_k](t_1,\ldots,t_n))^{I_\tr} &=&
    f_{\sigma_\I(i_1),\ldots,\sigma_\I(i_k)}\bigl(
      (t_1)^{I_\tr},\ldots,(t_n)^{I_\tr}
    \bigr)
    \\
    (m[i_1,\ldots,i_k](t_1,\ldots,t_n)@T)^{I_\tr} &=&
    \hat{m}(\sigma_\I(i_1),\ldots,\sigma_\I(i_k),
      (t_1)^{I_\tr},\ldots,(t_n)^{I_\tr},
      (T)^{I_\tr})
  \end{eqnarray*}
\end{definition}

\begin{example}
  Consider the meta-logic term $t := \mathsf{h}(\mout@a[i],\mathsf{k}[i])$
  and an interpretation $I_\tr$ with $D_\XT = [1;10]$ and $D_\I = [1;3]$
  (which might be relevant if we are considering traces of ten actions
  with three agents) such that $\sigma_\I(i)=2$ and $\hat{a}(2)=10$
  and $\hat{\mout}(10)=\mathsf{ok}$ (the message outputted at step 10
  is $\mathsf{ok}$ \adrien{I don't like this. It seems that macro can be interperted as the term we want at any timestamp.}).
  We then have $(t)^{I_\tr} = \mathsf{h}(\mathsf{ok},\mathsf{k}_2)$.
\end{example}

The reason why macros take only one timestamp argument is purely practical:
we have no use for more. At this point one might wonder why we separate
indices and actions given that they are interpreted similarly: it is again
purely practical, we will use them for different purposes, and we will need
less structure on indices than on actions, making reasoning easier on them
(one can simply compare indices, there is no ordering and no predecessor
operation on them).


\begin{definition}
  If $I_\tr$ is a meta-interpretation and $e\in D_\I$,
  $I_\tr[i\mapsto e]$ is the interpretation where $\sigma_\I$ is
  modified so that $\sigma_{\I}(i)=e$.
  We define the translation of a meta-logic formula $\phi$
  into the base logic \emph{term} $(\phi)^{I_\tr}$
  as follows:
  \begin{eqnarray*}
    (\phi\wedge\phi')^{I_\tr} &=& (\phi)^{I_\tr} \stackrel{.}{\wedge} (\phi')^{I_\tr}
    \quad \text{and similarly for other boolean connectives} \\
    (\forall i:\idx. \phi)^{I_\tr} &=&
    \stackrel{.}{\wedge}_{e\in D_\I} (\phi)^{I[i\mapsto e]} \\
    (\forall \tau:\timestamp. \phi)^{I_\tr} &=&
    \stackrel{.}{\wedge}_{e\in D_\XT} (\phi)^{I[\tau\mapsto e]} \\
    (\exists i:\idx. \phi)^{I_\tr} &=&
    \stackrel{.}{\vee}_{e\in D_\I} (\phi)^{I[i\mapsto e]} \\
    (\exists \tau:\timestamp. \phi)^{I_\tr} &=&
    \stackrel{.}{\vee}_{e\in D_\XT} (\phi)^{I[\tau\mapsto e]} \\
    (i=i')^{I_\tr} &=&
    \left\{\begin{array}{ll}
      \mathsf{true} & \text{if } \sigma_\I(i)=\sigma_\I(i') \\
      \mathsf{false} & \text{otherwise}
    \end{array}\right. \\
    (T = T')^{I_\tr} &=&
    \left\{\begin{array}{ll}
      \mathsf{true} & \text{if } (T)^{I_\tr} = (T')^{I_\tr} \\
      \mathsf{false} & \text{otherwise}
    \end{array}\right.
    \quad\text{ and similarly for $\leq$} \\
    (\happens(T))^{I_\tr} &=& \mathsf{true}
    \text{ if } (T)^{I_\tr} \in H_\XT
    \text{ and } \mathsf{false} \text{ otherwise}
    \\
    (t=t')^{I_\tr} &=& (t)^{I_\tr} \stackrel{.}{=} (t')^{I_\tr}
  \end{eqnarray*}
\end{definition}

\begin{definition}
  A formula $\phi$ of the meta-logic is said to be valid when,
  for any meta-interpretation $I_\tr$, the base logic formula
  $(\phi)^{I_\tr} \sim \mathsf{true}$ is valid.

  In other words, we have $\M,\sigma\models(\phi)^I\sim\mathsf{true}$
  i.e. the boolean term $(\phi)^{I_\tr}$ is true with overwhelming
  probability in any computational model $\M$ and for any interpretation
  $\sigma$ of the free message variables.\footnote{
  In the tool, we only allow universal quantification over messages, and
  only allow it at toplevel. The validity of such formulas is the same as
  when the variables are left free.}
\end{definition}


\begin{definition}
  A meta-logic formula $\phi$ is a
  logical consequence of a set $S$ of meta-logic formulas
  (noted $S \models \phi$)
  when
  $\M\models(\phi)^{I_\tr}\sim\mathsf{true}$ holds for any $\M$ and ${I_\tr}$ such that
  $\M\models(\psi)^{I_\tr}\sim\mathsf{true}$ holds for all $\psi\in S$.
  \adrien{This definition seems not necessary. We interpret meta-formula as (schemas of) first-order formulas. Hence entailment is exactly first-order logic entailment.}
\end{definition}


\section{Trace properties}

\subsection{Validity and logical consequence}

We have defined a translation from any meta-logic formula $\phi$ to a base logic
boolean term $\interp{\phi}^\TM$. We can now formally define when a protocol
satisfies some trace property $\phi$.  Intuitively, the meta-logic formula is
true when the associated term is true with overwhelming probability. This is
made formal in the following definitions.


\begin{definition}
  Given a protocol $P$, a formula $\phi$ of the meta-logic is said to be valid
  when, for any trace model $\TM$, the base logic formula
  $(\phi)^\TM \sim \true$ is valid.

  More generally, a meta-logic formula $\phi$ is a
  logical consequence of a set $S$ of base logic formulas
  (which might themselves be of the form $\interp{\psi}^\TM \sim \true$)
  when, for all $\TM$, $\Mo$ and $\sigma$ such that
  $\Mo,\sigma \models \psi$ for all $\psi\in S$,
  we have
  $\Mo,\sigma \models(\phi)^\TM\sim\mathsf{true}$.
\end{definition}

\begin{remark}
  In the tool, we only allow universal quantification over messages, and
  only allow it at toplevel. The validity of such formulas is the same as
  when the variables are left free.
\end{remark}

In practice, we want to verify that a security property expressed as a
meta-logic formula $\phi$ is satisfied in a class of trace models
$\TM$ and computational models $\Mo$: we will typically restrict computational
models so that cryptographic primitives satisfy some security assumption,
but we may also restrict trace models e.g.\ to force a condition on
abstract traces. In order to verify security in such a class of models,
we will identify axioms that hold in these models, and verify that our
security property is a logical consequence of these axioms.

\begin{remark}
  We have defined a notion of validity that implies that a formula must hold for all abstract traces. Abstract traces can contain actions that are not executable, and we can thus write formulas about outputs that may never be performed. In practice, we will often prove the validity of formulas of the form $\mexec@T \Rightarrow \phi$. This implies that the trace is executable, and allows us to reason over the conditions of the actions.
\end{remark}

\subsection{Sequent calculus}

In the tool, sequents come with an environment which explicitly
declares all the variables that might occur free in the sequent's formulas.
For simplicity, we do not include them here. We shall write $\vdash t:\Sort$
when $t$ is a term of sort $\Sort$ (which might be $\idx$ or $\timestamp$).

\begin{definition}
  A sequent $\Gamma \vdash \phi$ is composed of a set of meta-logic formulas
  $\Gamma$ and a meta-logic formula $\phi$.
  It is valid when the meta-logic formula
  $(\wedge\Gamma) \Rightarrow \phi$ is valid.
\end{definition}

\begin{proposition}
  The rules of \cref{fig:lk,fig:names} are sound: if the premisses are valid,
  then so is the conclusion.
\end{proposition}

\begin{figure}
  \begin{mathpar}
    \inferrule[Axiom]{~}{\Gamma,\phi\vdash\phi}
    \quad\quad
    \inferrule[Cut]{
      \Gamma \vdash \phi
      \quad
      \Gamma,\phi \vdash \psi
    }{
      \Gamma \vdash \psi
    }
  \end{mathpar}
  \begin{mathpar}
  \inferrule[${\wedge}$-L]{
    \Gamma,\phi,\phi'\vdash\psi
  }{
    \Gamma,\phi\wedge\phi'\vdash\psi
  }
  \quad\quad
  \inferrule[${\wedge}$-R]{
    \Gamma \vdash \phi
    \quad
    \Gamma \vdash \phi'
  }{
    \Gamma \vdash \phi\wedge\phi'
  }
  \end{mathpar}
  \begin{mathpar}
    \inferrule[${\lnot}$-R]{
      \Gamma, \phi \vdash \bot
    }{
      \Gamma \vdash \lnot\phi
    }
    \quad\quad
    \inferrule[${\lnot}$-L]{
      \Gamma \vdash \phi
    }{
      \Gamma,\lnot\phi \vdash \psi
    }
    \quad\quad
    \inferrule[Raa]{
      \Gamma, \lnot\phi \vdash \bot
    }{
      \Gamma \vdash \phi
    }
  \end{mathpar}
  \begin{center}
  \emph{other propositional rules of classical sequent calculus}
  \end{center}
  \begin{mathpar}
      \inferrule[${=}$-R]{~}{\Gamma \vdash t=t} \quad\quad
      \inferrule[${=}$-L]{
        (\Gamma\vdash\phi)\{x\mapsto t',x'\mapsto t\}
      }{
        t=t', \Gamma\{x\mapsto t,x'\mapsto t'\} \vdash
        \phi\{x\mapsto t,x'\mapsto t'\}
      }
  \end{mathpar}
  \begin{mathpar}
     \inferrule[$\forall$-L]{
       \Gamma,\phi\{x\mapsto t\} \vdash \psi
       \quad
       \vdash t:\Sort
     }{
       \Gamma,\forall x:\Sort.\phi \vdash \psi}
     \quad\quad
     \inferrule[$\forall$-R]{
       \Gamma \vdash \phi}{\Gamma \vdash \forall x:\Sort:\phi}
  \end{mathpar}
   \caption{Generic inference rules. These are the rules of classical
   first-order sequent calculus. In \textsc{$\forall$-R} we require that
   $x$ does not appear free in $\Gamma$.}
   \label{fig:lk}
\end{figure}

\begin{figure}
  \begin{mathpar}
  \inferrule{
    \mathsf{n}\neq\mathsf{m}
  }{
    \Gamma,\mathsf{n}[\vec i]=\mathsf{m}[\vec j]\vdash \phi
  }
  \quad\quad
  \inferrule{
    \Gamma, i_1=j_1, \ldots, i_k=j_k \vdash \phi
  }{
    \Gamma,\mathsf{n}[i_1,\ldots,i_k]=\mathsf{n}[j_1,\ldots,j_k]\vdash \phi
  }
  \end{mathpar}
  \begin{mathpar}
    \inferrule{~}{
      \Gamma,t=\mathsf{n}[\vec i] \vdash \phi
    }
  \end{mathpar}
  \caption{Inference rules for equalities on names.
  The last rule only applies when, for any trace model $\TM$,
  the term $\interp{t}^\TM$ does not contain any message variable and does
  not contain any occurrence of $\mathsf{n}_{\sigma_\I(\vec i)}$
  --- this condition will be over-approximated in our implementation.}
  \label{fig:names}
\end{figure}

\begin{figure}
  \begin{mathpar}
    \inferrule{
      \Gamma,t=\mathsf{n}[\vec {j_0}],
      \big(
        \bigvee_{\mathsf{m}[\vec j] \in t}
          \mathsf{m}[\vec j] = \mathsf{n}[\vec {j_0}]
      \big)
      \vee
      \big(
        \bigvee_{\tau \in t}
          \big(
            \bigvee_{A(\vec i) \in S, \mathsf{n}[\vec j] \in A(\vec i)}
              A(\vec i\{\vec j \mapsto \vec {j_0}\}) \leq \tau
          \big)
      \big)
      \vdash \phi
    }{
      \Gamma,t=\mathsf{n}[\vec {j_0}] \vdash \phi
    }
  \end{mathpar}
  \caption{Another inference rule for equalities on names.
  This rule only applies when, for any trace model $\TM$,
  the term $\interp{t}^\TM$ does not contain any message variable.
  If $\tau$ refers to an input macro, then the inequality over timestamps
  is strict.}
\end{figure}

\begin{proposition}
  The rules of \cref{fig:xor} are sound wrt.\ computational models
  where $\oplus$ is interpreted as exclusive or:
  if the premisses are valid in all such computational models,
  then so is the conclusion.
  \emph{Do we need to impose a condition on message lengths?}
\end{proposition}

\begin{figure}
  \begin{mathpar}
    \inferrule{~}{\Gamma \vdash t\oplus t = 0}
    \quad\quad
    \inferrule{~}{\Gamma \vdash t\oplus t' = t'\oplus t}
    \quad\quad
    \inferrule{~}{\Gamma \vdash t\oplus (t'\oplus t'') = (t\oplus t')\oplus t''}
  \end{mathpar}
  \begin{mathpar}
    \inferrule{
      (\Gamma \vdash \phi)\{x\mapsto \mathsf{m}[\vec j]\}
    }{
      (\Gamma \vdash \phi)\{x\mapsto t\oplus\mathsf{n}[\vec i]\}
    }
  \end{mathpar}
  \caption{Inference rules for exclusive or.
  The tool does not follow these rules closely, but uses a more high-level
  congruence closure algorithm modulo xor. The last rule is not implementated
  at all and is here mostly for illustration purposes, and is subject to
  the condition that, for any trace model $\TM$,
  $\interp{\Gamma\vdash\phi}^\TM$ does not contain instances of the names
  $\mathsf{n}$ and $\mathsf{m}$ and does not contain message variables.
  This condition would be over-approximated in an implementation.}
  \label{fig:xor}
\end{figure}

\newcommand{\eufcma}{\textsc{EUF-CMA}}
\begin{proposition}
Let $P$ be a protocol.  In any computational model where $h$ is interpreted as a $\eufcma$ keyed hash-function, the following rule is valid for all term $t,m$ and name $sk$ such that all ocurences of $sk$ in $t$ or $P$ are of the form $h(\_,sk)$:
  \begin{mathpar}
    \inferrule{
      (\Gamma,
      \big(\bigvee_{h(x,sk) \in t} m=x \big)
    }{
      (\Gamma, t=h(m,sk) \vdash \phi)
    }
  \end{mathpar}
    \begin{mathpar}
    \inferrule{
      (\Gamma,
      \big(\bigvee_{A(\vec{i}) \in S} \bigvee_{h(x,sk) \in o_{A(\vec{i})} } m=x \wedge A(\vec{i}) \leq \tau\big)
      ,\vdash \phi)
    }{
      (\Gamma, \minp@\tau=h(m,sk) \vdash \phi)
    }
    \end{mathpar}
  \end{proposition}
\subsection{Trace axioms}

We now identify some valid formulas, which can thus be used as axioms
when trying to derive a trace property.

\begin{proposition}
  The following formulas are valid w.r.t.\ the protocol:
  \begin{itemize}
    \item for any $\alpha \leq \beta$ (partial ordering of the symbolic actions imposed by the protocol), $\alpha \leq \beta$.

    \item
      $\forall \tau.~ \pre(\tau)<\tau \wedge
      \forall \tau'.~ \pre(\tau)\leq\tau'<\tau \Rightarrow
      \tau'=\pre(\tau)$;
    \item for any $a\neq b\in\Act$,
      $\forall \ov{i},\ov{j}, a(\ov{i})\neq b(\ov{j})$;
    \item for any $a\in\Act$,
      $\forall \ov{i},\ov{j}, a(\ov{i})=a(\ov{j}) \Rightarrow \ov{i}=\ov{j}$.
      \item $\forall \tau.~ \mexec@\tau \Rightarrow \forall \tau'.~\tau'\leq \tau \Rightarrow \mcond@\tau$
  \end{itemize}
\end{proposition}


The first axiom holds because for any trace model, we asks that $\leq_\XT$ is compatible with the ordering $\leq$ over symbolic actions.
As $\pre$ is always interpreted as the natural notion of predecessor over some discrete sequence, the second axiom is natural. The next two axioms hold for all trace models, because we ask that the ordered sequence of concrete actions produced by $\leq_\XT$ represent an abstract trace of the protocol. This yields the injectivity axiom. The last axiom holds naturally by the definition of $\mexec$.

%%% Local Variables:
%%% mode: latex
%%% TeX-master: "main"
%%% End:


\section{Indistinguishability}

\newcommand{\pair}[1]{\langle #1 \rangle}

% We now define indistinguishability of processes, and how we could verify it.
% We make a distinction between symbolic and observable traces:
% symbolic ones are sequences of action names as in \cref{def:trace},
% while observable traces are sequences of pairs $\pair{c_i,c_o}$.
% Intuitively, such a pair describes the input and output channels of an
% action; several actions might have the same input and output channels.
% We write $A : \pair{c_i,c_o}$ when action $A$ inputs on channel $c_i$ and
% outputs on channel $c_o$.

% For convenience, we talk of processes, though in reality the definitions
% deal with abstract systems described by sets of actions.

\newcommand{\fold}{\mathsf{fold}}

% \begin{definition}
%   Given an observable trace $t$ and a process $P$,
%   we define $\fold(P,t)$ as the frame describing all possible
%   executions of $P$ along $t$:
%   \begin{eqnarray*}
%     \fold(P,\epsilon) &=& \epsilon \\
%     \fold(P,(c_i,c_o).t') &=&
%     \mathsf{if}_{A:(c_i,c_o)}
%     \ldots
%   \end{eqnarray*}
% \end{definition}

% \begin{definition}
%   Two processes $P$ and $Q$ are indistinguishable when,
%   for all observable traces $t$,
%   for all computational model $\Mo$,
%   we have
%   $\Mo \models \fold(P,t) \sim \fold(Q,t)$.
%   This itself means that no attacker can distinguish,
%   with non-negligible probability, between the left- and right-hand sides.
% \end{definition}

% This notion of equivalence should coincide with the usual computational
% indistinguishability for bounded processes. In the unbounded case, it
% is restrictive to quantify on traces in this way
% and ask for indistinguishability only for each trace.

\subsection{Straightforward diff-equivalence}

We define the diff-equivalence of a process, by asking the equivalence of the projected frames for all possible traces.

\begin{definition}
  \label{def:process-equiv}
  A bi-process $P$ is diff-equivalent when,
  for any trace model $\TM$, the formula

  \[\stackrel{.}{\wedge}_{v\in D_\XT} (\mframe@\tau^L)^{\TM[\tau\mapsto v]} \sim (\mframe@\tau^R)^{\TM[\tau \mapsto v]}\]
  is valid.
  \adrien{Why not just take the last action?}
\end{definition}
\charlie{Notice that we could also extend $\phi^\TM$ to formulas containing the $\sim$ symbol, and consider the formula $\forall \tau. \mframe@\tau^L \sim \mframe@\tau^R$.}
Note that this equivalence cannot hold if there exists a trace
whose probability of execution significantly differs between the two
projections of the bi-process.
Hence this imposes a form of synchronization on the execution of
conditionals on the two sides of bi-processes.
The interest of imposing this artificial constraint is that we
can stop considering executions where one process goes to $\mythen$
branch while the other goes to its $\myelse$ branch. \adrien{The constraint is not artificial. Indeed, since the actions are visible, the adversary sees whether the process is going right or left. There is no loss of generality there.}
\begin{lemma}
  A bi-process $P$ is diff-equivalent if,
  for any trace model $\TM$, the formula

  \[\stackrel{.}{\wedge}_{v\in D_\XT} (\pair{\mframe@\tau^L,\mexec@\tau^L})^{\TM[\tau\mapsto v]} \sim (\pair{\mframe@\tau^R, \mexec@\tau^R} )^{\TM[\tau \mapsto v]}\]
  is valid.
  \adrien{This just complicated to show than what Definition~\ref{def:process-equiv} requires. Maybe there is a typo?}
\end{lemma}

\newcommand{\In}{\mathsf{in}}
\newcommand{\Out}{\mathsf{out}}

\begin{example}
  Consider the bi-process
  $\In(c,x).\myif x=\diff{0}{1} \mythen \Out(c,n) \myelse \Out(c,m)$
  where $n$ and $m$ are arbitrary, possibly equal names.
  Its two projections are indistinguishable, but the
  bi-process is not diff-equivalent.
  \adrien{I do not agree. If you write the process using actions (as we do), then it is equivalent. When you are translating from the pi-process to an representation using actions, you cannot change the visible actions (or this is not a sound translation).}
  Indeed we have
  $\myif g()=0 \mythen \Out(c,n) \not\sim
  \myif g()=1 \mythen \Out(c,m)$: the attacker can simply choose
  $g()=0$ to distinguish the two sides.
  Our bi-process can however easily be
  rewritten into a diff-equivalent process, e.g. by pushing the conditional
  inside the output.

  If we modify our bi-process into
  $\In(c,x).\myif x=\diff{0}{1} \mythen \Out(c,n) \myelse \Out(c,0)$
  then the two projections become distinguishable.
  The attack is obtained with an execution
  where one process outputs a name while the other outputs $0$. Such
  desynchronized executions are not taken into account with diff-equivalence,
  but diff-equivalence still fails due to the desynchronized condition,
  as before.
\end{example}

In the next examples, we omit the $\myelse$ branch when it consists of a null
process. In these examples, there is a coincidence between diff-equivalence
and indistinguishability, because observable actions coincide with symbolic
actions.

\begin{example} \label{ex:negl}
  Consider the bi-process
  $\In(c,x).\myif x=\diff{n}{m} \mythen \Out(c,\ok)$.
  It is diff-equivalent, and its projections are
  indistinguishable as expected.
  In the bi-process, the condition $x=\diff{n}{m}$ does not pass
  with the same inputs on the left and right, but it passes with
  the same negligible probability.
\end{example}

\begin{example} \label{ex:sync}
  Consider
  $\In(c,x).\myif x=(\diff{n}{m})_0 \mythen \Out(c,n)$
  where $(t)_0$ denotes the first bit of $t$.
  This bi-process is not diff-equivalent because
  $\myif x=(n)_0 \mythen n \not\sim \myif x=(m)_0 \mythen n$, and
  the two projections are distinguishable for the same
  reason: the attacker sends $0$;
  on the left he receives with probability $1\over 2$ a bitstring whose
  first bit is $0$;
  on the right process he receives with probably only $1\over 4$
  a bitstring whose last bit is $0$.
  If we change $\Out(c,n)$ into $\Out(c,\ok)$,
  we have indistinguishable processes and diff-equivalence holds.
\end{example}

\begin{example} \label{ex:problem}
  Consider $\Out(c,\diff{n}{m}).
  \In(c,x).
  \myif x=\diff{n}{m} \mythen \Out(c,\ok)$.
  The projections are observationally equivalent and diff-equivalence
  holds -- in fact they are $\alpha$-equivalent.
\end{example}


\subsection{Reasoning about equivalences}

We extend the sequent calculus, to also reason about diff-equivalence.

\begin{definition}
  A sequent $\Gamma \vdash t_L \sim t_R $ is composed of a set of indistinguishability $\Gamma$, and sequences of terms $t_L,t_R$.
  It is valid when, for any trace model $\TM$, for any evaluations $\sigma_\XT',\sigma_I'$ of the free variables in $t_L,t_R$ to well typed elements in $\TM$, the formula $ (\wedge \Gamma)^{\TM'} \Rightarrow (t_L)^{\TM'} \sim  (t_R)^{\TM'}$, where $\TM'=\TM \cup \sigma_\XT'\cup \sigma_I'$.
\end{definition}

This definition of sequent calculus can of course be used to reason about diff-equivalence.
\begin{lemma}
  A bi-process is diff-equivalent if and only if,
  \[ \emptyset \vdash  \mframe^L@\tau \sim \mframe^R@\tau\]
  Equivalently, is it diff-equivalent if and only if:
  \[\mframe^L@\pre(\tau) \sim \mframe^R@\pre(\tau) \vdash \mframe^L@\tau \sim \mframe^R@\tau\]
\end{lemma}
\begin{proof}We prove the first equivalence, which is essentially an unfolding of definitions.

  \[
    \begin{array}{l@{~}l}
      $P$\text{ is diff-equivalent} & \Leftrightarrow \text{for all } \TM,\ \stackrel{.}{\wedge}_{v\in D_\XT} (\mframe@\tau^L)^{\TM[\tau\mapsto v]} \sim (\mframe@\tau^R)^{\TM[\tau \mapsto v]}\text{ is valid}\\
      & \Leftrightarrow \text{for all } \TM\text{ and } v\in D_\XT,\ (\mframe@\tau^L)^{\TM[\tau\mapsto v]} \sim (\mframe@\tau^R)^{\TM[\tau \mapsto v]}\text{ is valid}\\
      & \Leftrightarrow \text{for all } \TM\text{ and } v\in D_\XT,\ \true \stackrel{.}{\Rightarrow} (\mframe@\tau^L)^{\TM[\tau\mapsto v]} \sim (\mframe@\tau^R)^{\TM[\tau \mapsto v]}\text{ is valid}\\
      & \Leftrightarrow  \emptyset \vdash  \mframe^L@\tau \sim \mframe@\tau^R\text{ is valid}\\

    \end{array}
  \]

  The second equivalence is a direct induction on the length of the traces.
\end{proof}

Adding the execution condition to the sequent is also a valid proof technique.
\begin{lemma}
  A bi-process is diff-equivalent if,
  \[ \emptyset \vdash  \pair{\mframe^L@\tau,\mexec@\tau^L} \sim \pair{\mframe@\tau^R,\mexec@\tau^R}\]
  \adrien{It is harder to prove the formula above than to prove $\emptyset \vdash  \mframe^L@\tau \sim \mframe^R@\tau$. Maybe there is a typo?}
\end{lemma}
We provide in \cref{fig:lk-ind} a set of sound rules for this second sequent calculus.

\begin{figure}
  \begin{mathpar}
    \inferrule[Expand]{\Gamma \vdash t_L \sim t_R
      \quad
    }{\Gamma \cup \phi \vdash t_L \sim t_R}
    \quad\quad
    \inferrule[Cut]{
      \Gamma \vdash t_L \sim t_R
      \quad
      \Gamma \vdash t_R \sim t_S
    }{\Gamma \vdash t_L \sim t_S}
  \end{mathpar}
  \begin{mathpar}
    \inferrule[Subst]{
      \Gamma \cup \{\EQ(t_1,t_2)\sim \true \}  \vdash t^L \sim t^R
    }{
      \Gamma \cup \{\EQ(t_1,t_2)\sim \true \} \vdash t^L[ t_1 / t_2] \sim  t^R[ t_1 / t_2]
    }

  \end{mathpar}

  \begin{mathpar}
    \inferrule[\myif-reach]{
      \phi \vdash \false
    }{
      \Gamma  \vdash \myif \phi \mythen t_L \sim  \myif \phi \mythen t_R
    }
    \quad\quad
    \inferrule[\myif-equiv]{
      \phi \vdash \psi \Leftrightarrow \psi'
    }{
      \Gamma \vdash \myif  \phi \wedge \psi \mythen t \sim \myif \phi \wedge \psi' \mythen t
    }
  \end{mathpar}
  \begin{mathpar}
    \inferrule[${\lnot}$-R]{
      \Gamma \vdash \false \sim \true
    }{
      \Gamma \vdash t^L \sim t^R
    }
    \quad\quad
    \inferrule[fresh]{
      n,m \not \in \Gamma
    }{
      \Gamma \vdash n \sim m
    }
    \quad\quad
    \inferrule[\myif-weak]{
      \Gamma \vdash \phi, t_L \sim \phi, t_R
    }{
      \Gamma \vdash \myif  \phi \mythen t_L \sim \myif \phi \mythen t_R
    }
  \end{mathpar}
  \begin{mathpar}
    \inferrule[F-Cut]{
      \Gamma \vdash \phi
      \quad
      \Gamma \cup \phi \vdash t^L \sim t^R
    }{
      \Gamma \vdash t^L \sim t^R
    }
    \quad\quad
    \inferrule[Dup]{
      \Gamma \vdash t^L,u \sim t^R,v
    }{
      \Gamma \vdash t^L,u,u \sim t^R,v,v
    }
  \end{mathpar}


  \caption{Generic inference rules for indistinguishability}
  \label{fig:lk-ind}
\end{figure}
\begin{lemma}
  The rules presented in Figure~\ref{fig:lk-ind} are sound.


\end{lemma}

\paragraph{Other rules.}

\Cref{fig:fresh,fig:prf,fig:xor} presents the rules for Fresh, PRF and XOR tactics.
We use the following notations:
\begin{itemize}
\item $\Gamma \vdash u$ stands for $\Gamma^L \sim \Gamma^R \vdash u^L \sim u^R$
\item $A \in S$ stands for every action in the system (or protocol)
\item $A(\vec i)^L$ represents the left projection of meta-logic bi-terms and bi-formulas describing the action $A(\vec k)$ (outputs, updates and conditions)
\item $k(\_) \sqsubseteq_{\h(\_,\cdot)} u$ means that the indexed key $k$ appears only in key positions in $u$
\item indices $\vec i$ in $A(\vec i)$ are chosen fresh with relation to the appropriate environment (i.e. indices appearing in $u, C, t, k, \vec {j_0}$)
\end{itemize}

\begin{figure}[h]
  \begin{mathpar}
    \inferrule[Fresh]{
      \Gamma \vdash u, C[\myif \phi_L \wedge \phi_R \mythen 0 \myelse n(\vec {j_0})]
    }{
      \Gamma \vdash u, C[n(\vec {j_0})]
    }
  \end{mathpar}
  \begin{mathpar}
    \phi_L = \big( \displaystyle\bigwedge_{n^L(\vec {j}) \in u^L,C^L} \vec {j} \neq {\vec {j_0}^L} \big)
    \wedge
    \big(
    \displaystyle\bigwedge_{A \in S}
    \forall \vec i \
    \big(\displaystyle\bigvee_{\tau \in u^L,C^L} A(\vec i) \leq \tau \big)
    \Rightarrow
    \big(\displaystyle\bigwedge_{n^L(\vec j) \in A(\vec i)^L} \vec j \neq {\vec {j_0}^L} \big)
    \big)
  \end{mathpar}
  \begin{mathpar}
    \phi_R = ... \ \text{(similar, replacing $L$ by $R$)}
  \end{mathpar}
  \caption{Fresh rule.\solene{Not yet implemented with the context $C$.}}
  \label{fig:fresh}
\end{figure}

\begin{figure}[h]
  \begin{mathpar}
    \inferrule[PRF]{
      \Gamma \vdash u, C[\myif \diff{\phi_L}{\phi_R} \mythen n \myelse \h(t,k(\vec {j_0}))]
      \quad \quad
      k(\_) \sqsubseteq_{\h(\_,\cdot)} u,C,A(\vec i)
    }{
      \Gamma \vdash u, C[\h(t,k(\vec {j_0}))]
    }
  \end{mathpar}
  \begin{mathpar}
    \phi_L = \big( \displaystyle\bigwedge_{\h(m,k(\vec j)) \in u^L,C^L,t^L} (\vec j = {\vec {j_0}}^L \Rightarrow t^L \neq m) \big)
    \wedge
    \big(
    \displaystyle\bigwedge_{A \in S}
    \forall \vec i \
    \big(\displaystyle\bigvee_{\tau \in u^L,C^L,t^L} A(\vec i) < \tau \big)
    \Rightarrow
    \big(\displaystyle\bigwedge_{\h(m,k(\vec j)) \in A(\vec i)^L} (\vec j = {\vec {j_0}}^L \Rightarrow t^L \neq m) \big)
    \big)
  \end{mathpar}
  \begin{mathpar}
    \phi_R = ... \ \text{(similar, replacing $L$ by $R$)}
  \end{mathpar}
  \caption{PRF rule.\solene{Not yet implemented with the context $C$.}}
  \label{fig:prf}
\end{figure}

\begin{figure}[h]
  \begin{mathpar}
    \inferrule[XOR-ind]{
      \Gamma \vdash u, C[\myif \phi_L \wedge \phi_R \mythen m \myelse t \oplus n(\vec {j_0})]
    }{
      \Gamma \vdash u, C[t \oplus n(\vec {j_0})]
    }
  \end{mathpar}
  \begin{mathpar}
    \phi_L = \big( \displaystyle\bigwedge_{n^L(\vec j) \in u^L,C^L,t^L} \vec j \neq {\vec {j_0}}^L \big)
    \wedge
    \big(
    \displaystyle\bigwedge_{A \in S}
    \forall \vec i \
    \big(\displaystyle\bigvee_{\tau \in u^L,C^L,t^L} A(\vec i) \leq \tau \big)
    \Rightarrow
    \big(\displaystyle\bigwedge_{n^L(\vec j) \in A(\vec i)^L} \vec j \neq {\vec {j_0}}^L \big)
    \big)
  \end{mathpar}
  \begin{mathpar}
    \phi_R = ... \ \text{(similar, replacing $L$ by $R$)}
  \end{mathpar}
  \caption{XOR-ind rule.\solene{Not yet implemented with the context $C$.}}
  \label{fig:xor}
\end{figure}

\begin{figure}[h]
  \begin{mathpar}
    \inferrule[FA-DUP]{
      \inferrule{
        \Gamma \vdash u,
        \mframe @ \pre(A(\vec i)),
        \mexec @ \pre(A(\vec i))
      }{
        \Gamma \vdash u,
        \mframe @ \pre(A(\vec i)),
        \myif \mexec @ \pre(A(\vec i)) \mythen \phi_{h} \myelse \bot
      }
    }{
      \Gamma \vdash u,
      \mframe @ \pre(A(\vec i)),
      \mexec @ \pre(A(\vec i)) \wedge \phi_{h}
    }
  \end{mathpar}

  We ask that $\phi_h \in H_{\{\pre(A(\vec i))\}}$ where, for any set of
  timestamps $T$, $H_T$ is the least set of formulas and messages
  closed under function application, boolean connectives, and the
  following rules:
  $$ \inferrule{
    B(\vec j) \in T
    \quad
    \phi \in H_{T\cup\{C(\vec {k})\}}
  }{
    (\forall \vec k.~ C(\vec k)\leq B(\vec j) \Rightarrow \phi) \in H_T
  }
  \quad\quad
  \inferrule{
    B(\vec j) \in T
    \quad
    \phi \in H_{T\cup\{C(\vec k)\}}
  }{
    (\exists \vec k.~ C(\vec k)\leq B(\vec j) \wedge \phi) \in H_T
  }
  $$
  $$\inferrule{B(\vec j) \in T}{
    \minp @ B(\vec j) \in H_T
  }\quad\quad
  \inferrule{ }{
    \minp @ A(\vec i) \in H_T
  }\quad\quad
  \inferrule{B(\vec j) \in T}{
    \mout @ B(\vec j) \in H_T
  }$$
  $$
  \inferrule{\phi \in H_T \quad B(\vec j) \in T}{
    (\myif \mexec @ B(\vec j) \mythen \phi \myelse \psi) \in H_T}
  $$
  \caption{FA-DUP rule.
  }
  \label{fig:fadup}
\end{figure}

\clearpage
\section{Archives}
\subsection{A proof technique}

Diff-equivalence is usually proved by induction and case analysis on
the timestamp. Even cases where the left and right actions are locally
identical are not trivial: it may be e.g.\ that the same name is outputted
by the action on both sides, but that each side has previously released
different information on that name.

To prove diff-equivalence, it can be interesting to prove that:
$$ \phi^L_\tr, \phi^R_\tr \vdash \phi^L_\alpha \Leftrightarrow \phi^R_\alpha $$

Then, in the induction step, we can directly replace the previous conditions by the same one.


This condition is only used to help for proving diff-equivalence, it is not necessary.
The gap comes from the fact that we are requiring conditions to be
synchronized for all random samplings.

\begin{example}
  With the bi-process of \cref{ex:negl} we would have to prove
  $\vdash g() = n \Leftrightarrow g() = m$ and
  $g()=n, g()=m \vdash \ok \sim \ok$, both of which hold.
  With the bi-process of \cref{ex:sync} we would have to prove
  $\vdash g() = (n)_0 \Leftrightarrow g() = (m)_0$ which does not hold.
\end{example}

\begin{example} \label{ex:indep}
  This proof technique does not work for \cref{ex:problem}.
  The same problem appears with the Basic-Hash protocol, even if we work around
  the problem described in \cref{sec:refined-diff}, we won't be able to show
  that conditionals are synchronized.  In the simple case of the trace
  $T(i,j).R(k,i,j)$ we have
  on the single-session side
  $$\pi_2(g_2(\pair{n_T(i,j),h(n_T(i,j),k'(i,j))})) =
  h(\pi_1(g_2(\ldots)),k'(i,j))$$
  and we would like this to imply (in the meta-logic)
  the same equality with $k(i)$ instead of $k'(i,j)$.
  This implication does not hold with overwhelming probability in all
  computational models: as in \cref{ex:problem}, $g_2(x)$ could be the second
  projection of $x$ with its first bit changed to $0$; if the hash is PRF,
  there should be a probability of roughly $1 \over 4$ that this leaves
  the hash unchanged with $k'(i,j)$ but not with $k(i)$.
\end{example}

\section{Outdated example : a signed DDH key exchange}
\charlie{abus de notations dans cette partie}

We brieffly show how one can prove the security of a signed DDH key exchange. The protocol in pi-calculus is provided in Figure~\ref{fig:signed_ddh} and the run of an honnest execution in Figure~\ref{fig:dh_ke}. This example is a simplified instance of classical key-exchange security. Notably, we assume that identities are already fixed.

\begin{figure}
  % \setlength{\belowcaptionskip}{-15pt}
  \setmsckeyword{} \drawframe{no}
  \setmscscale{0.9}
  \begin{center}
    \begin{msc}{}
      \setlength{\instwidth}{0\mscunit}
      \setlength{\instdist}{7cm}
      \setlength{\topheaddist}{0cm}
      \declinst{initiator}{
        \begin{tabular}[c]{c}
          \textsc{A} \\
          \colorbox{gray}{{\;\; $sk_A,a_i$\;\;}}
        \end{tabular}}{}

      \declinst{receiver}{
        \begin{tabular}[c]{c}
          \textsc{B} \\
          \colorbox{gray}{{\;\;  $sk_B,b_i$ \;\;}}
        \end{tabular}}{}

      \nextlevel[-1]
      \mess{$\mysign(g^{a_i},sk_A)$}{initiator}{receiver}
      \nextlevel[1.5]
      \mess{$\mysign(<g^{a_i},g^{b_i}>,sk_B)$}{receiver}{initiator}
      \nextlevel[1.5]
      \mess{$\mysign(<g^{a_i},g^{b_i}>,sk_A)$}{initiator}{receiver}




    \end{msc}
  \end{center}
  \caption{Diffie Hellman key exchange}\label{fig:dh_ke}
\end{figure}

\begin{figure}
  \[
    \begin{array}{cc}
      \begin{array}[t]{l@{~}l}
        A_i := & \aout{\mysign(g^{a_i},sk_A)} : \alpha_1; \\
        &\ain{x}; \\
        & \myif \mycsign(x,pk(sk_B)) \\
        & ~ \wedge \pi_1(\mygetmess(x))=g^{a_i}  \mythen \\
        & \quad \aout{\mysign(\mygetmess(x),sk_A) } : \alpha_2; \\
        & \quad \myfind j \mysuchthat g^{b_j} = \pi_2(\mygetmess(x)) \\
        & \qquad \aout{\diff{\pi_2(\mygetmess(x))^{a_i}}{k_{i,j}}} : \alpha_3\\
        & \quad \myelse \\
        & \qquad \aout{\diff{\pi_2(\mygetmess(x))^{a_i}}{\bot}}  : \alpha_4 \\
        & \myelse \\
        & \bot
      \end{array}
      &
      \begin{array}[t]{l@{~}l}
        B_i := &\ain{x}; \\
        & \myif \mycsign(x,pk(sk_A)) \mythen \\
        & \quad \aout{\mysign(<\mygetmess(x), g^{b_i}>,sk_B)} : \beta_1; \\
        & \quad \ain{y}; \\
        & \quad \myif \mycsign(y,pk(sk_A))\\
        & \quad ~ \wedge \mygetmess(y) = <\mygetmess(x), g^{b_i}> \mythen \\
        & \quad \quad \myfind j \mysuchthat g^{a_j} = \mygetmess(x) \\
        & \quad \qquad \aout{\diff{\mygetmess(x)^{b_i}}{k_{j,i}}} : \beta_2 \\
        & \quad \quad \myelse \\
        & \quad \qquad \aout{\diff{\mygetmess(x)^{b_i}}{\bot}} : \beta_3 \\
        & \quad \myelse \\
        & \quad \bot \\
        & \myelse \\
        & \bot
      \end{array}
    \end{array}
  \]
  \label{fig:signed_ddh}
  \caption{A signed DDH key exchange}
\end{figure}

We outline the proof of the fact that $!_i A_i \| B_i$ is diff-equivalent. There are four actions with choices in the output, thus, we have to show that, for all trace $\tr$, for all $i,j$:
\begin{enumerate}
\item $\begin{array}[t]{l}
    \phi_\tr,  \mycsign(x,pk(sk_B)), \mygetmess(x)=<g^{a_i}, g^{b_j}>, \mouts_\tr^L \sim \mouts_\tr^R,  \\
    \quad \vDash \mouts_\tr^L, \pi_2(\mygetmess(x))^{a_i} \sim \mouts_\tr^R, k_{i,j}
  \end{array}
  $ ($\alpha_3$)

\item $\begin{array}[t]{l}
    \phi_\tr,  \mycsign(x,pk(sk_B)),  \not \exists j. \pi_2(mygetmess(x))= g^{b_j},  \mouts_\tr^R \sim \mouts_\tr^L \\
    \quad \vDash \mouts_\tr^L, \pi_2(\mygetmess(x))^{a_i} \sim \mouts_\tr^R, \bot
  \end{array} $ ($\alpha_4$)
\item $\begin{array}[t]{l}
    \phi_\tr,  \mycsign(y,pk(sk_A)), \mycsign(x,pk(sk_A)), \mygetmess(y)=<g^{a_j}, g^{b_i}>,  \mouts_\tr^R \sim \mouts_\tr^L \\ \quad \vDash \mouts_\tr^L, \pi_2(\mygetmess(x))^{b_i} \sim \mouts_\tr^R, k_{j,i}
  \end{array}$ ($\beta_2$)
\item $\begin{array}[t]{l}
    \phi_\tr,  \mycsign(y,pk(sk_A)), \mycsign(x,pk(sk_A)),  \not \exists j. \pi_2(mygetmess(x))= g^{b_j},  \mouts_\tr^R \sim \mouts_\tr^L\\
    \vDash \mouts_\tr^L, \pi_2(\mygetmess(x))^{b_i} \sim \mouts_\tr^R, \bot
  \end{array}$ ($\beta_3$)
\end{enumerate}

Regarding goal $(2)$ and $(4)$, we remark that it is a case where $\Gamma \vdash \false$. Indeed, for $(2)$ applying EUFCMA yields that there exists $j$ such that $x = \mysign(<g^{a_i},g^{b_j}>,sk_B)$ which is in contradiction with  $\not \exists j. \pi_2(mygetmess(x))= g^{b_j}$.

Regarding goals $(1)$ and $(3)$, we mainly use DDH. To this end, we first use EUFMCA, to ensure that we have a matching conversation between the two sessions, and then use DDH. \charlie{je détail pas, ça prend du temps de formaliser proprement DDH vis à vis des actions, et je pense pas que ce soit l'objectif actuel}




%%% Local Variables:
%%% mode: latex
%%% TeX-master: "main"
%%% End:


\end{document}

%%% Local Variables:
%%% mode: latex
%%% TeX-master: t
%%% End:
