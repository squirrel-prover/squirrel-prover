\documentclass[a4paper]{article}

\usepackage{lipsum}
\usepackage[normalem]{ulem}
\usepackage{mathtools}
\usepackage{mathpartir}
\usepackage{stmaryrd}
\usepackage{amsthm}
\usepackage{float}
\usepackage{wrapfig}
\usepackage{framed}
\usepackage{xcolor}
\usepackage{textcomp}
\usepackage{xspace}
\usepackage{amsmath,amssymb,amsfonts}
\usepackage[T1]{fontenc}
\usepackage[utf8]{inputenc}
\usepackage{cite}
\usepackage{algorithmic}
\usepackage{graphicx}
\usepackage{placeins}
\usepackage[margin=2cm]{geometry}
\usepackage{multicol}

\usepackage{url}
\usepackage{tikz}
\usepackage{proof}
\usepackage{hyperref}
\usepackage[capitalize,nameinlink,noabbrev]{cleveref}
\usepackage{enumerate}

\setlength{\ULdepth}{0.15em}

\usetikzlibrary{backgrounds}
\usetikzlibrary{intersections}
\usetikzlibrary{scopes}
\usetikzlibrary{calc}
\usetikzlibrary{decorations.pathreplacing}
\usetikzlibrary{decorations.pathmorphing,shapes}
\usetikzlibrary{backgrounds}
\usetikzlibrary{shapes.misc}

\newtheorem{assumption}{Assumption}
\newtheorem*{theorem*}{Theorem}
\newtheorem*{lemma*}{Lemma}
\newtheorem*{proposition*}{Proposition}

\newtheorem{definition}{Definition}
\newtheorem{proposition}{Proposition}
\newtheorem{theorem}{Theorem}
\newtheorem{lemma}{Lemma}
\newtheorem{corollary}{Corollary}

\theoremstyle{remark}
\newtheorem{remark}{Remark}

\newcommand{\todo}[1]{\textcolor{red}{(\textbf{TD:} #1)}}
\newcommand{\toadd}[1]{\textcolor{blue}{(\textbf{TA:} #1)}}

\newcommand{\charlie}[1]{\textcolor{cyan}{(\textbf{Charlie:} #1)}}

\newcommand{\true}{\textsf{true}}
\newcommand{\false}{\textsf{false}}

\newcommand{\pvec}[1]{\vec{#1}\mkern2mu\vphantom{#1}}

\newcommand{\sfr}{\textsf{r}}
\newcommand{\sfx}{\textsf{x}}
\newcommand{\sfy}{\textsf{y}}
\newcommand{\sfz}{\textsf{z}}
\newcommand{\sfb}{\textsf{b}}

\newcommand{\bc}{\textsf{BC}}
\newcommand{\ra}{\rightarrow}
\newcommand{\mdp}{\mathbin{\|}}

\newcommand{\mdef}{\;\;\mathbin{:=}\;\;}

\newcommand{\cexists}{\exists}
\newcommand{\cforall}{\forall}

\newcommand{\judge}[3]{#1 \mdp #2 \vdash #3}
\newcommand{\subst}[3]{#1\{#2 \mapsto #3\}}
\newcommand{\substs}[2]{#1\{#2\}}

\newcommand{\thetap}{{\theta'}}
\newcommand{\thetapp}{{\theta''}}
\newcommand{\thetao}{{\theta_0}}
\newcommand{\thetat}{{\theta_1}}
\newcommand{\psip}{{\psi'}}
\newcommand{\deltap}{{\delta'}}
\newcommand{\gammap}{{\gamma'}}

\newcommand{\valpha}{\pvec{\alpha}}
\newcommand{\valphap}{\pvec{\alpha}'}
\newcommand{\valphao}{{\pvec{\alpha}_0}}
\newcommand{\valphat}{{\pvec{\alpha}_1}}
\newcommand{\valphatt}{{\pvec{\alpha}_2}}
\newcommand{\vbeta}{\pvec{\beta}}
\newcommand{\vbetap}{\pvec{\beta}'}
\newcommand{\vgamma}{\pvec{\gamma}}
\newcommand{\vgammap}{\pvec{\gamma}'}
\newcommand{\vdelta}{\pvec{\delta}}
\newcommand{\vdeltap}{\pvec{\delta}'}

\newcommand{\wf}{\textsf{well-formed}}

\newcommand{\env}{\textsf{E}}
\newcommand{\envp}{\textsf{E'}}

\newcommand{\pcnstr}{\mathcal{P}}
\newcommand{\cnstr}{\textsf{C}}

\newcommand{\act}{{\textsf{a}}}
\newcommand{\aset}{\mathcal{S}}

\newcommand{\decl}[3]{\left(#1 , #2 : #3\right)}
\newcommand{\decls}{\mathcal{D}}

\newcommand{\pvars}{\textsf{p-vars}}
\newcommand{\pvtype}[2]{#1 : #2}
\newcommand{\facts}[3]{\pvtype{#1}{#2} \mid #3}

\newcommand{\centail}[3]{#1 \mid #2 \vdash_{\textsf{c}} #3}
% Rules
\newcommand{\defunroll}[1]{\textsf{unroll}($#1$)}
\newcommand{\pvintro}{\textsf{pv-intro}}
\newcommand{\fintro}{\textsf{f-intro}}

\newcommand{\clstrength}{\textsf{c-left-strength}}
\newcommand{\crstrength}{\textsf{c-right-strength}}
\newcommand{\cempty}{\textsf{c-empty}}
\newcommand{\cdisj}{\textsf{c-disj}}

\newcommand{\ceelim}{\textsf{c-}$\exists$\textsf{-elim}}

\newcommand{\landelim}{$\wedge$\textsf{-left-elim}}
\newcommand{\lorelim}{$\vee$\textsf{-left-elim}}

\newcommand{\randelim}{$\wedge$\textsf{-right-elim}}


\newcommand{\apply}{\textsf{apply}}
\newcommand{\requ}{\textsf{r-equ}}


%%% Local Variables:
%%% mode: latex
%%% TeX-master: "main"
%%% End:


\begin{document}
\title{Meta-Logic for BC}


\author{David Baelde, Adrien Koutsos}

\maketitle

\section{A Logic for protocols}

\subsection{Syntax of the logic}
We denote $\F$ a set of function symbols (used to model encryptions, 
pairs,\dots). $\N$ denotes an 
infinite set of names (used to model the randomness used in a 
protocol), and $\I$ denotes an infinite set of indexes (used to talk 
about multiple copies of the same agent). For 
each name $n\in \N$ and sequence of indices $\ov{i} \in \I$, 
$n(\ov{i})$ refers to a fresh unique name in $\N$. We denotes $\X$ an 
infinite set of identifiers, used for variables.

We finally denote $\States$ a set of identifiers used to model 
mutable 
states, which can also be parametrized by sequences of indices, 
$\Actions$ a set of identifiers used to model actions 
(which 
will be used 
to model an atomic step of a protocol), and $\Timestamps$ a set of 
identifiers for timestamps in a trace. 

The syntax of the Meta-BC logic is then given in \cref{fig:syntax}.

\begin{figure}
$$
\begin{array}[t]{ccc}
    \begin{array}{rcl}
    f \in \F & & \text{function symbol}\\
    i \in \I  & & \text{index}\\

    x \in \X & & \text{variable}\\
    \end{array}
&     \begin{array}{rcll}
\tau &:=& \tau \in \Timestamps  & \text{timestamp}\\
& \mid & a \in \Actions & \text{action} \\
& \mid & \suc(\tau) & \text{action} \\
\end{array}
   &
    \begin{array}{rcll}
    
    n &:=& n \in \N & \text{name}\\
    &\mid& n(i_1,\dots,i_n) & \text{name copy} \\
    \\
    
    s&:=& s \in \States & \text{state}\\
    &\mid& s(i_1,\dots,i_n) & \text{state copy}\\
    
    \end{array}
\end{array}  
$$

        

$$
\begin{array}{c}
     \begin{array}{rcll}
    \T &  := & \mess \mid \timestamp \mid \idx \mid \bool & \text{ 
        type of variables} \\
    
      \\
   
    t & := & n &\text{a name } \\
           & \mid & s@\tau &\text{value of a state}\\
    & \mid & x  & \text{a variable} \\
    & \mid & \mout@\tau & \text{value of some output} \\
    & \mid & \minp@\tau & \text{value of some input} \\
    & \mid & f(t_1,\dots,t_n) &\text{function application}\\
       \end{array}
\\
\\
   \begin{array}{rcll}
   \atom & := & t=t'  
 & \text{atomic proposition over terms } \\
  &\mid & \tau=\tau' \mid \tau \leq \tau' &  \text{atomic proposition 
  over timestamps } \\
  &\mid & i=i'  &  \text{atomic proposition 
    over index } \\
  &\mid & a= a'\mid  \happens(a)  &  \text{atomic proposition 
    over actions } \\
    \end{array}
\\
\\
     \begin{array}{rcll}
    \phi & ::= &  \true \mid \false \mid \phi \wedge \phi' \mid  \phi 
    \vee \phi' \mid   \phi \Rightarrow \phi'\mid \neg \phi \mid 
    \forall x : \T.\ \phi \mid \exists x:\T.\ \phi \mid \atom & 
    \text{ 
    first order formula} \\
    \end{array}
    
\end{array}        
    $$
    \caption{Syntax of first order formulas.}\label{fig:syntax}
\end{figure}


Terms $t$ may represent messages of a protocol, where given a trace, 
i.e a sequence of actions or timestamps, $\mout@\tau$ and 
$\minp@\tau$ models the output and input of the protocol at this 
point in the trace. We will only consider well typed formulas, i.e 
formulas where all variables (which are typed) are used accordingly 
to there type (e.g, $s@x$ is ill-typed if $x$ is of type $\mess$).


Timestamps are ordered, and it is possible to consider the successor 
of a timestamp.  Terms and indices are not ordered.

\subsection{Protocols definitions}



We will model protocols as a set of possible actions available to the 
attacker. An action essentially models a step of the protocol, where 
the attacker provides some inputs, some conditions are checked, some 
updates are performed, and finally an output is performed.


\newpage
\section{Old stuff}



\paragraph{Facts}
Facts are schemas of $\bc$ formulas indexed by path constraints:
\[
  \phi \mdef
  \cforall_\theta\, \valpha.\;
  \psi
  \ra
  \bigvee_{i}
  \cexists_\gammai \vbetai. \psi_i
\]

\paragraph{Environments}
\[
  \env \mdef
  \aset \mid
  \pcnstr \mid
  \decls
  \qquad
  \text{ where }
  \qquad
  \decls \mdef
  \left\{
    \decl{t_\sfx}
    {t_\epsilon,(t_\sfy^\act)_{\act \in \aset}}
  \right\}
\]
where:
\begin{itemize}
\item $\aset$ is the set of symbolic actions.
\item $\pcnstr$ is a set of path constraints.
\item $\decls$ is a set of inductive term declarations.
\end{itemize}
\paragraph{Judgments}
\[
  \ejudge{\env}{\facts{\valpha}{\theta}{\Gamma}}{\phi}
\]
where:
\begin{itemize}
\item $\env$ is the environment.
\item $\pvtype{\valpha}{\theta}$ declares a set of constrained path variables.
\item $\Gamma$ is a set of facts.
\item $\phi$ is the goal.
\end{itemize}

\begin{remark}
  $\env$ will be invariant during derivations. We omit it when there is no ambiguity.
\end{remark}

\paragraph{Standard Introduction Rules}
\begin{mathpar}
  \inferrule[\rraintro]{
    \judge{\facts{\valpha}{\theta}{
        \Gamma \cup \{\psi_0\}}}
    {
      \cexists_\gammap \vdelta. \psi
    }
  }{
    \judge{\facts{\valpha}{\theta}{\Gamma}}
    {
      \psi_0
      \ra
      \cexists_\gammap \vdelta. \psi
    }
  }

  \inferrule[\andlintro]{
    \judge{\facts{\valpha}{\theta}{\Gamma \cup \{ \psi \} \cup \{\psi'\}}}
    {\phi}
  }{
    \judge{\facts{\valpha}{\theta}{\Gamma \cup \{ \psi \wedge \psi'\}}}
    {\phi}
  }

  \inferrule[\orlintro]{
    \judge{\facts{\valpha}{\theta}{\Gamma \cup \{ \psi \}}}
    {\phi}\\
    \judge{\facts{\valpha}{\theta}{\Gamma \cup \{\psi'\}}}
    {\phi}
  }{
    \judge{\facts{\valpha}{\theta}{\Gamma \cup \{ \psi \vee \psi'\}}}
    {\phi}
  }

  \inferrule[\andrintro]{
    \judge{\facts{\valpha}{\theta}{\Gamma}}{\psi}\\
    \judge{\facts{\valpha}{\theta}{\Gamma}}{\psi'}
  }{
    \judge{\facts{\valpha}{\theta}{\Gamma}}{\psi \wedge \psi'}
  }

  \inferrule[\orrintro-1]{
    \judge{\facts{\valpha}{\theta}{\Gamma}}{\psi}
  }{
    \judge{\facts{\valpha}{\theta}{\Gamma}}{\psi \vee \psi'}
  }

  \inferrule[\orrintro-2]{
    \judge{\facts{\valpha}{\theta}{\Gamma}}{\psi'}
  }{
    \judge{\facts{\valpha}{\theta}{\Gamma}}{\psi \vee \psi'}
  }
  
  \inferrule[\toprintro]{
  }{
    \judge{\facts{\valpha}{\theta}{\Gamma}}{\top}
  }
\end{mathpar}

\paragraph{Other Rules}
\begin{mathpar}
  \inferrule[\absurd]{
    \bot \in \Gamma
  }{
    \judge{\facts{\valpha}{\theta}{\Gamma}}{\phi}
  }

  \inferrule[\axiom]{
    \psi \in \Gamma
  }{
    \judge{\facts{\valpha}{\theta}{\Gamma}}{\psi}
  }  

\end{mathpar}

\paragraph{Modal Introduction Rules}
\begin{mathpar}
  \inferrule[\grintro]{
    \judge{\facts{\valpha,\vbeta}{\theta\wedge\gamma}{\Gamma}}
    { \phi }\\
    \valpha \cap \vbeta = \emptyset
  }{
    \judge{\facts{\valpha}{\theta}{\Gamma}}
    {
      \cforall_\gamma\, \vbeta.\;
      \phi
    }
  }

  \inferrule[\printro]{
    \nu : \vbeta \mapsto \valpha\\
    \judge{\facts{\valpha}{\theta}{\Gamma}}{\psi\nu}\\
    \centail{\theta}{\gamma\nu}
  }{
    \judge{\facts{\valpha}{\theta}{\Gamma}}{\cexists_\gamma \vbeta. \psi}
  }
\end{mathpar}

\paragraph{Path Constraints Rules}
\begin{mathpar}
  \inferrule[\clweaken]{
    \judge
    {\facts{\valpha}{\gamma}{\Gamma}}{\phi}\\
    \centail{\theta}{\gamma}
  }{
    \judge{\facts{\valpha}{\theta}{\Gamma}}{\phi}
  }

  \inferrule[\crstrengthen]{
    \judge{\facts{\valpha}{\theta}{\Gamma}}{\cexists_\delta \vbeta. \phi}\\
    \centail{\delta}{\gamma}
  }{
    \judge{\facts{\valpha}{\theta}{\Gamma}}{\cexists_\gamma \vbeta. \phi}
  }
  
  \inferrule[\cempty]{
    \centail{\theta}{\bot}
  }{
    \judge{\facts{\valpha}{\theta}{\Gamma}}{
      \phi
    }
  }

  \inferrule[\cdisj]{
    \judge{\facts{\valpha}{\theta}{\Gamma}}{\phi}\\
    \judge{\facts{\valpha}{\gamma}{\Gamma}}{\phi}
  }{
    \judge{\facts{\valpha}{\theta \vee \gamma}{\Gamma}}{\phi}
  }
\end{mathpar}

\paragraph{Rules}
\begin{mathpar}
  \inferrule[\defunroll{t_\tau}]{
    \decl{t_\sfx}
    {t_\epsilon,(t_\sfy^\act)_{\act \in \aset}} \in \decls\\
    \tau \in \valpha\\
    \tauo \not \in \valpha\\\\
    \judge
    {\facts{\valpha}
      {\theta\wedge\tau = \epsilon}
      {\Gamma \wedge t_\tau = t_\epsilon}}
    {\phi}\\
    \left(
      \judge
      {\facts{\valpha,\tauo}
        {\theta\wedge\tauo = \tpred{\tau}\wedge \tauo = \act}
        {\Gamma \wedge t_\tau = t_\tauo^\act}}
      {\phi}
    \right)_{\act \in \aset}
  }{
    \judge{\facts{\valpha}{\theta}{\Gamma}}{\phi}
  }

  \inferrule[\requ]{
    \judge{\facts{\valpha}{\theta}{\Gamma}}{\psi[s]}\\
    \judge{\facts{\valpha}
      {\theta}{\Gamma}}
    {s = t}
  }{
    \judge{\facts{\valpha}{\theta}{\Gamma}}{
      \psi[t]
    }
  }
  
  \inferrule[\papply]{
    \cexists_\gamma \vbeta. \psi
    \in \Gamma\\
    \valpha \cap \vbeta = \emptyset\\
    \judge{\facts{\valpha, \vbeta}{\theta \wedge \gamma}
      {\Gamma \cup \{\psi\}}}{\phi}
  }{
    \judge{\facts{\valpha}{\theta}{\Gamma}}{\phi}
  }

  \inferrule[\apply]{
    (\cforall_\gamma\, \vbeta.\;
    \psi
    \ra
    \phi_0)
    \in \Gamma\\
    \nu : \vbeta \mapsto \valpha\\
    \centail{\theta}{\gamma\nu}\\
    \judge{\facts{\valpha}{\theta}{\Gamma}}{\psi\nu}\\
    \judge{\facts{\valpha}{\theta}
      {\Gamma \cup \{\phi_0\nu\}}}{\phi}
  }{\judge{\facts{\valpha}{\theta}{\Gamma}}{\phi}}

  \inferrule[\induc]{
    \phi' \equiv
    \cforall_{\gammao} \tauo.\phi\\
    \gammao \equiv
    \subst{\gamma}{\tau}{\tauo}\wedge\tauo < \tau\\
    \tau,\tauo \not \in \valpha\\
    \judge
    {\facts{\valpha,\tau}{\theta\wedge\gamma}{\Gamma\cup\{\phi'\}}}
    {\phi}
  }{
    \judge
    {\facts{\valpha}{\theta}{\Gamma}}
    {\cforall_\gamma \tau.\phi}
  }
\end{mathpar}

\paragraph{Superpositions of $\clweaken$ or $\crstrengthen$ with other rules}
\begin{itemize}
\item $\mrule{\clweaken}$ with $\mrule{\printro}$:
  \[
    \inferrule[\printrogen]{
      \nu : \vbeta \mapsto \valpha\\
      \judge{\facts{\valpha}{\theta \wedge (\gamma\nu)}{\Gamma}}{\psi\nu}\\
      \judge{\facts{\valpha}{\theta \wedge \neg(\gamma\nu)}{\Gamma}}{\bot}
    }{
      \judge{\facts{\valpha}{\theta}{\Gamma}}{\cexists_\gamma \vbeta. \psi}
    }
  \]

\item $\mrule{\crstrengthen}$ with $\mrule{\apply}$ and $\mrule{\absurd}$:
  \[
    \inferrule[\botapply]{
      (
      \cforall_\gamma\, \vbeta.\;
      \psi
      \ra
      \bot
      ) \in \Gamma\\
      \nu : \vbeta \mapsto \valpha\\
      \judge{\facts{\valpha}{\theta \wedge (\gamma\nu)}{\Gamma}}{\psi\nu}\\
      \judge{\facts{\valpha}{\theta \wedge \neg(\gamma\nu)}{\Gamma}}{\phi}
    }{\judge{\facts{\valpha}{\theta}{\Gamma}}{\phi}}
  \]

\item $\mrule{\crstrengthen}$ with $\mrule{\apply}$ (general case):
  \[
    \inferrule[\botapply]{
      (
      \cforall_\gamma\, \vbeta.\;
      \psi
      \ra
      \phio
      ) \in \Gamma\\
      \nu : \vbeta \mapsto \valpha\\
      \judge{\facts{\valpha}{\theta \wedge (\gamma\nu)}{\Gamma}}{\psi\nu}\\
      \judge{\facts{\valpha}{\theta \wedge (\gamma\nu)}
        {\Gamma \cup \{ \phio\nu\}}}
      {\phi}\\
      \judge{\facts{\valpha}{\theta \wedge \neg(\gamma\nu)}{\Gamma}}{\phi}
    }{\judge{\facts{\valpha}{\theta}{\Gamma}}{\phi}}
  \]

\end{itemize}
\end{document}

%%% Local Variables:
%%% mode: latex
%%% TeX-master: t
%%% End:
